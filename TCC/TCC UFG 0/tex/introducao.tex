% Contextualizar:
O concreto é o material de construção mais utilizado, sendo considerado a segunda \wander{substância}substancia mais utilizada do mundo, perdendo apenas para água \cite{Gagg2014}, sua utilização se estende até nas chamadas Obra de Arte Especial (OAE), estruturas cujo objetivo é a transposição de obstáculos, como pontes, avenidas, viadutos e túneis. De acordo com Departamento Nacional de Infraestrutura de Transportes (DNIT) (2006) \wander{a palavra apud deve estar em itálico, qualquer palavra de outra língua deve estar em itálico}apud Mendes (2009), existem 73.000 quilômetros de rodovias pavimentadas e não pavimentadas do modal rodoviário brasileiro, contendo dentro dessas, 5.600 pontes. Em 2011 um relatório do Tribunal de Contas da União (Relatório TC 003.134/2011-3) apontou em uma auditoria que o valor estimado das OAE's são da ordem 13 bilhões de reais e esses estão distribuídos em cerca de 4.500 pontes e viadutos na malha federal. Já em 2015 o DNIT contabilizou somente sob sua responsabilidade um total de 5114 OAE’s.

Para estudar tais estruturas de concreto, em especifico suas condições referentes à durabilidade e ampliação de vida útil, foi nomeado de SHM (\textit{Structural Health Monitoring}) o campo referente a tais estudos, que vêm evoluindo a um nível em que como conta \citeonline{de2011durabiliidade}, hoje a vida útil de uma estrutura seja avaliada de acordo com seus anos de vida e não mais com os critérios qualitativos de adequação da estrutura ou grau de exposição. Alguns desses estudos tem como foco, técnicas aplicadas em OAE`s, cujo objetivo é alertar aos pesquisadores ou responsáveis da estrutura sobre a real condição em que se encontra \cite{inaudi2009structural}. Um exemplo desses estudos é a norma NBR 15575, que é utilizada para verificação de durabilidade para sistemas estruturais de edificações habitacionais.

\wander{Lucas veja que não tem muito sentido a frase depois de tais estudos...Vc fala que vem evoluindo mais não mostra nada da evolução. Se vc ler atentamente vai ver que as frases não tem conexão nenhuma}

\wander{Veja minha recomendação: O SHM (\textit{Structural Health Monitoring}) é uam das ténicas computacionais que tem como objetivo avaliar o comportamento real da estrutura e avaliar a sua qualidade como produto de engenharia. Essa técnica tem como objetivo principal informar o estado real em que a condição se encontra.}

\wander{Ai lucas aqui acho que o caminho da evolução quer eu acho que vc quis dizer é falar que antes e até hj essa inspeção é muito manual e custosa. E agora pode ser feitas por métodos computacionais.}



\section{Problema}

Normas como essa são necessárias pois por melhor que seja o concreto, atualmente ainda é comum que se manifestem patologias, como fissuras (Rachaduras e buracos) ou exposição da sustentação metálica da estruturas (Armadura da estrutura), seja por conta de causas naturais ou por erro humano em sua criação. Por conta disso é de suma importância que ocorra um monitoramento constante sobre essas estruturas no geral, até mesmo as consideradas saudáveis. 
O comum é que empresas tenham uma equipe em especifico responsável pelo monitoramento constante, entretanto, o custo humano pode ser diminuído com a utilização de diferentes tecnologias, com isso em mente, uma das opções bem aceitas pela comunidade do SHM como alternativa ao trabalho manual humano é a utilização de sensores para avaliação do estado real de uma estrutura, porém os sensores tem uma série de desvantagens, como a manutenção constante, difícil substituição, e ser necessário um conjunto enorme de sensores para cobrir o espaço de uma OAE. Como alternativa, outra técnica que está ganhando espaço é a de utilizar reconhecimento de imagens, que consiste no reconhecimento de padrões de manifestação patológicos a partir de imagens da estrutura.


\section{Objetivos}

Na área de inteligencia artificial, a área de visão computacional vem ganhando cada vez melhores resultados utilizando reconhecimento de padrões e segundo \citeonline{jain2000statistical} podendo ser aplicado em diversas áreas como: Mineração de dados (\textit{data mining}); Análise de imagens; Análise de texto; Inspeção visual para automação industrial; Busca e classificação em base de dados multimídia; Reconhecimento biométrico, incluindo faces, íris ou impressões digitais.

Dentro deste vasto campo da inteligência artificial essa proposta para projeto de pesquisa tem como objetivo utilizar as técnicas de conhecimento de visão computacional, em específico utilizar redes neurais convolucionais aplicadas em análise de imagens para detectar falhas de sistema estruturais do tipo OAE através de imagens das mesmas. \wander{As falhas estudadas neste trabalho tratam-se das fissuras no concreto.}

\section{Metodologia}

Para alcançar tal proposta, pretende se basear no artigos de \citeonline{spencer2019advances}, \citeonline{ham2016visual}, \citeonline{gomes2008some} e \citeonline{narazaki2021synthetic} que aplicaram tal conhecimento à problemática acima, utilizando imagens de câmeras autônomas ou unidades aéreas como drones para realizar o monitoramento das estruturas, realizando uma captura de imagem ou vídeo para ser usado como base em uma inspeção utilizando redes neurais convolucionais para detectar patologias nas estruturas. Todos os autores em questão conseguiram resultados satisfatórios, o que mostra a eficácia em seu modelo de operação.

\section{Resultados}

Os resultados obtidos foram

\section{Organização do texto}

O texto é organizado da seguinte maneira: