% Contextualizar:
O concreto é o material de construção mais utilizado, sendo considerado a segunda substância mais utilizada do mundo, perdendo apenas para água \cite{Gagg2014}, sua utilização se estende até as chamadas \sigla{OAE}{Obra de Arte Especiais}, estruturas cujo objetivo é a transposição de obstáculos, como pontes, avenidas, viadutos e túneis. 
De acordo com o Departamento Nacional de Infraestrutura de Transportes (DNIT) (2006), \textit{apud} Mendes (2009), existem 73.000 quilômetros de rodovias pavimentadas e não pavimentadas do modal rodoviário brasileiro, contendo dentro dessas, aproximadamente 5.600 pontes. 
Em 2011 um relatório do Tribunal de Contas da União (Relatório TC 003.134/2011-3) apontou em uma auditoria que o valor estimado das OAE's são da ordem 13 bilhões de reais e esses estão distribuídos em cerca de 4.500 pontes e viadutos na malha federal. Já em 2015 o DNIT contabilizou somente sob sua responsabilidade um total de 5.114 OAE’s.

Para estudar tais estruturas de concreto, em específico suas condições referentes à durabilidade e ampliação de vida útil, foi desenvolvido um campo chamado de \sigla{SHM}{Structural Health Monitoring} \cite{de2011durabiliidade}.
O SHM utiliza principalmente de técnicas computacionais que tem como objetivo avaliar o comportamento real da estrutura e avaliar a sua qualidade como produto de engenharia. 
Essas técnicas tem como objetivo principal informar o estado real em que a estrutura se encontra \cite{inaudi2009structural}.


\section{Problema}

Aplicações do SHM dentro do Brasil são especificadas em normas, como a norma NBR 15575, que é utilizada para verificação de durabilidade para sistemas estruturais de edificações habitacionais.
Normas como essa são necessárias pois por melhor que seja o concreto, atualmente ainda é comum que se manifestem patologias, como fissuras (rachaduras e buracos) ou exposição da sustentação metálica da estruturas (armadura da estrutura), sejam causadas por ações naturais ou por erro humano em sua criação \cite{afonso2021}. 
Dessa forma é de suma importância que ocorra um monitoramento constante das estruturas no geral, até mesmo das que são consideradas saudáveis \cite{statera}.

Por via de regra espera-se que empresas tenham uma equipe específica responsável pelo monitoramento constante de tais estruturas \cite{statera}.
Essa responsabilidade é de alto custo operacional, logo, é sugerido substituir o custo humano por técnicas computacionais que constam no SHM \cite{Liu2002}.

Dentre essas tecnologias, destacam-se a utilização de sensores e heurísticas aplicadas em visão computacional. 
Todavia, essas tecnologias possuem certas desvantagens que desencorajam sua utilização, como custo exorbitante de instalação e manutenção para os sensores e inconsistência de resultado para heurísticas \cite{Zhuang2022}.

Assim sendo, é sugerido utilizar técnicas de aprendizagem profunda que vem obtendo cada vez melhores resultados.
Segundo \citeonline{jain2000statistical} já há bons resultados em diversas áreas como: mineração de dados (\textit{data mining}); análise de imagens; análise de texto; inspeção visual para automação industrial; busca e classificação em base de dados multimídia; reconhecimento biométrico, incluindo faces, íris ou impressões digitais.

\section{Objetivos}
O objetivo deste trabalho é utilizar técnicas de aprendizagem profundo em união com visão computacional para o reconhecimento de padrões de manifestação patológicas a partir de imagens da estrutura.
Para tal, utiliza-se de redes neurais convolucionais aplicadas em análise de imagens para reconhecer padrões de falhas em sistemas estruturais e classificar estes em estruturas saudáveis ou que apresentem  patologias.

\section{Objetivos específicos}
Os objetivos específicos são:

\begin{itemize}
    \item Explorar os problemas que ocorrem em OAE's, como fissuras e exposição de armadura;
    \item Levantamento das bases de dados de imagens de fissuras;
    \item Levantamento dos modelos de aprendizado profundo que melhor se encaixam ao problema;
    \item Validação dos modelos nas bases de dados de imagens de fissuras em OAE's;
    \item Análise da correlação entre as bases de dados de imagens;
    \item Análise dos resultados.
\end{itemize}


\section{Metodologia}

O método utilizado consiste em utilizar quatro bases de dados indicadas por \citeonline{zoubir2021crack} e um subconjunto da união destas para o treinamento de duas arquiteturas de redes neurais convolucionais: \sigla{VGG }{\textit{Visual Geometry Group}} 16 \cite{simonyan2014very} e ResNet \cite{he2016deep}.
Cada arquitetura pode ser utilizada com e sem transferência de aprendizado.

O processo para utilização dessas bases de dados pode ser resumido em duas etapas principais. 
Primeiro, foi feita a busca e a compreensão dos artigos citados por \citeonline{zoubir2021crack}, o que permitiu encontrar as bases de dados online e fazer o \textit{download} delas localmente. 
Em seguida, cada base foi organizada em duas pastas, correspondentes a cada rótulo utilizado. 
Como os autores já disponibilizaram as bases de dados com a rotulação e o pré-processamento necessários, não foi necessário realizar esses procedimentos novamente.

Para a aplicação dos modelos de redes neurais convolucionais, cada uma dessas bases é dividida em duas partes: 90\% para o treinamento e 10\% para os testes. 
Durante o treinamento, a técnica de validação estratificada é utilizada com 10 grupos de divisão, aplicando os 90\% da base de treinamento. 
A acurácia e o \textit{loss} são calculadas para cada grupo, e em seguida, a média, a variância e o desvio padrão são determinados a partir desses valores.
Em seguida, os resultados da validação cruzada são analisados para avaliar a integridade, confiabilidade e robustez dos modelos.

Caso essas métricas sejam satisfatórias, o melhor modelo de cada arquitetura é selecionado para uma bateria de testes. 
Os primeiros testes são realizados com os 10\% mencionados anteriormente, ou seja, os 10\%. 
Por fim, o modelo selecionado é aplicado a todas as outras bases de dados para avaliar a correlação entre elas e determinar qual é a melhor base para treinar um modelo.
A partir desses testes, métricas como acurácia, precisão, sensibilidade, especificidade e $F_{1}$-Score são calculados e utilizados para analise dos modelos e das bases.


\section{Contribuições para a área de pesquisa}

No geral, os resultados que utilizaram modelos de redes neurais convolucionais com transferência de aprendizado obtiveram desempenhos satisfatórios, com acurácias acima de 90\% e valores de $F_{1}-Score$ acima de 80\% para maioria dos casos.
Isso contando os resultados da validação cruzada estratificada e os testes realizados sobre a mesma base em que o modelo foi treinado.

Sobre esses resultados, os modelos VGG16 com transferência de aprendizagem, Densenet e Resnet apresentaram valores médios de $F_{1}-Score$ de 92,93\%, 91,11\% e 91,12\%, respectivamente.
Logo, o modelo VGG16 teve os melhores resultados.

Posteriormente, foram realizadas inspeções visuais de forma minuciosa das bases de dados para argumentar sobre os resultados encontrados.
Em seguida, experimentos foram realizados para avaliar o desempenho de modelos treinados em uma base de dados e testados em outras. 
Os resultados complementam  as análises visuais e mostraram que o uso de um subconjunto que abrange as imagens de todas as bases de dados é a base mais eficaz para treinamento. 
Nesses experimentos, o modelo Resnet foi o que obteve melhor desempenho com $F_{1}-Score$ de 96,44\%, indicando maior capacidade de abstração e generalização de dados em comparação com os outros modelos testados. 


Em suma, os resultados obtidos nesses experimentos são de grande relevância para a escolha de modelos e métodos de treinamento em redes convolucionais. 
Ademais, graças a esses experimentos, foi possível identificar qual base de dados apresentou os melhores resultados quando aplicada em todas as imagens. 
Essas informações são valiosas para orientar futuros trabalhos na área de processamento de imagens e aprendizado de máquina, contribuindo para o aprimoramento de técnicas e algoritmos utilizados em aplicações práticas.
Além disso, com o uso dessas técnicas avançadas de processamento de imagens e aprendizado de máquina, é possível reduzir significativamente o tempo e os custos envolvidos na inspeção e manutenção dessas estruturas, garantindo a segurança e a integridade das mesmas. 
Portanto, esses resultados têm implicações diretas na engenharia civil, auxiliando na manutenção e preservação de infraestruturas.

\section{Organização do texto}

O texto é organizado da seguinte maneira: Este primeiro capítulo apresenta a introdução do projeto. 
No \autoref{chapter:concreto} são apresentados os conceitos de estruturas de concreto e patologia, e sua problemática. 
No \autoref{chapter:deep_learn} é explicado sobre aprendizagem de máquina, desde seu conjunto maior até chegar em redes neurais convolucionais que é o foco desta monografia.

O \autoref{chapter:correlatos} apresenta trabalhos recentes na literatura que empregam redes neurais de forma correlata ao tema desta monografia.
No \autoref{chapter:metodologia} é dissertado a metodologia de como os experimentos que geram os resultados finais foram realizados.
No \autoref{chapter:resultados} são apresentados os resultados.
Por fim, o \autoref{chapter:conclusao} apresenta as conclusões da pesquisa.