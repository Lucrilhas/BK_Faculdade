% ------------------------------------------------------------------------
% ------------------------------------------------------------------------
% UFGRC: Modelo de Trabalho Acadêmico em conformidade com 
% ABNT NBR 14724:2011: Informação e documentação - Trabalhos acadêmicos -
% Apresentação
% ------------------------------------------------------------------------
% ------------------------------------------------------------------------

% Opções: 
%   Tipo do trabalho     = tcc1/tcc2
%   Situação do trabalho = pre-defesa/pos-defesa
% -- opções do pacote babel --
% Idioma padrão = brazil
	%english,			% idioma adicional para hifenização
	%brazil				% o último idioma é o PRINCIPAL do documento
\documentclass[tcc2, pos-defesa, english, brazil]{packages/ufgrc}

% ---------------------------------------------------------------------------
% Pacotes Opcionais
% ---------------------------------------------------------------------------
\usepackage{rotating}           % Usado para rotacionar o texto
\usepackage[all,knot,arc,import,poly]{xy}   % Pacote para desenhos gráficos
% Este pacote pode conflitar com outros pacotes gráficos como o ``pictex''
% Então é necessário usar apenas um dos pacotes conflitantes
\newcommand{\VerbL}{0.52\textwidth}
\newcommand{\LatL}{0.42\textwidth}
\newcommand\wander[1]{{\color{red}#1}} % Comando correção Wanderley
% ---------------------------------------------------------------------------



% ---
% Informações de dados para CAPA e FOLHA DE ROSTO
% ---
% Tanto na capa quanto nas folhas de rosto apenas a primeira letra da primeira palavra (ou nomes próprios) devem estar em letra maiúscula, todas as demais devem ser em letra minúscula.
\titulo{Aplicação de técnicas de visão computacional para detecção de fissuras em obras de arte especiais}
\autor[Cruvinel, L. E. A]{Lucas Elias de Andrade Cruvinel}
\genero{M} % Gênero do autor (M = Masculino / F = Feminino)
\orientador[Orientadora]{Prof.$^a$ Dr.$^a$}{Núbia Rosa da Silva}
\coorientador{Prof. Dr.}{Wanderlei Malaquias Pereira Junior}
\data{11}{11}{2022} % Data da defesa
% ---

% Membros da banca examinadora
% - O primeiro membro será automaticamente o orientador
% - Caso haja coorientador, este será o segundo membro
% Nome dos demais membros e suas instituições
\membrobanca{Fulano de Tal}{Instituição do Fulano de Tal}
\membrobanca{Ciclano de Tal}{Instituição do Ciclano de Tal}

% ---
% RESUMOS
% ---

% Resumo em PORTUGUÊS
% conter no máximo 500 palavras
% conter no mínimo 1 e no máximo 5 palavras-chave (obrigatoriamente separadas por vírgula)
\textoresumo[brazil]{
% Contexto
 O material de construção mais utilizado do mundo, o concreto, ainda possui problemas como fissuras e exposições de armadura que podem acontecer por conta da ação do tempo ou do ambiente em que se encontra, não sendo humanamente possível prever tudo o que acontecerá com o mesmo para que se prepare na hora de sua construção inicial. 
 %
 % lacuna
 Por conta disso se faz necessário que durante toda sua vida útil aconteça um monitoramento sobre o mesmo, o que é custoso e trabalhoso caso seja feito de forma manual ainda mais quando se trata de estruturas como Obra de Arte Especial onde a acessibilidade humana é mínima. 
 %
 % Objetivo
 A partir dessa problemática, este projeto de pesquisa tem como objetivo treinar um modelo computacional de rede neural convolucional para que através de imagens, seja possível classificar concreto como íntegro ou com necessidade de manutenção.
 %
 % Metodologia
 Para tal, será necessário a implementação de uma rede neural convolucional utilizando a base de dados de \citeonline{maguire2018sdnet2018} para treina-la até que se torne suficiente para resolver o problema. 
 %
 % Resultados esperados
 %Até o final desse projeto se busca ter a implementação completa dessa rede neural convolucional e com resultados aceitos pela comunidade de engenheiros estruturais.
}{Inspeção e monitoramento de Concreto, Fissuração e exposição da armadura, Rede neural convolucional}


% resumo em INGLÊS
% conter no máximo 500 palavras
% conter no mínimo 1 e no máximo 5 palavras-chave (obrigatoriamente separadas por vírgula)
\textoresumo[english]{
% Contexto
 O material de construção mais utilizado do mundo, o concreto, ainda possui problemas como fissuras e exposições de armadura que podem acontecer por conta da ação do tempo ou do ambiente em que se encontra, não sendo humanamente possível prever tudo o que acontecerá com o mesmo para que se prepare na hora de sua construção inicial. 
 %
 % lacuna
 Por conta disso se faz necessário que durante toda sua vida útil aconteça um monitoramento sobre o mesmo, o que é custoso e trabalhoso caso seja feito de forma manual ainda mais quando se trata de estruturas como Obra de Arte Especial onde a acessibilidade humana é mínima. 
 %
 % Objetivo
 A partir dessa problemática, este projeto de pesquisa tem como objetivo treinar um modelo computacional de rede neural convolucional para que através de imagens, seja possível classificar concreto como íntegro ou com necessidade de manutenção.
 %
 % Metodologia
 Para tal, será necessário a implementação de uma rede neural convolucional utilizando a base de dados de \citeonline{maguire2018sdnet2018} para treina-la até que se torne suficiente para resolver o problema. 
 %
 % Resultados esperados
 %Até o final desse projeto se busca ter a implementação completa dessa rede neural convolucional e com resultados aceitos pela comunidade de engenheiros estruturais.
}{Inspeção e monitoramento de Concreto, Fissuração e exposição da armadura, Rede neural convolucional}
    
% ---
% Configurações de aparência do PDF final
% ---
\hypersetup{
	colorlinks=true     % false: boxed links; true: colored links
}
% --- 

% ----------------------------------------------------------
% ELEMENTOS PRÉ-TEXTUAIS
% ----------------------------------------------------------

% Inserir a ficha catalográfica
%\incluifichacatalografica*{tex/pre-textual/fichaCatalografica.pdf}
\incluifichacatalografica

% DEDICATÓRIA / AGRADECIMENTO / EPÍGRAFE
%\textodedicatoria*{tex/pre-textual/dedicatoria}
%\textoagradecimentos*{tex/pre-textual/agradecimentos}
%\textoepigrafe*{tex/pre-textual/epigrafe}

% Inclui a lista de figuras
%\incluilistadefiguras

% Inclui a lista de tabelas
%\incluilistadetabelas

% Inclui a lista de quadros
%\incluilistadequadros

% Inclui a lista de algoritmos
%\incluilistadealgoritmos

% Inclui a lista de códigos
%\incluilistadecodigos

% Inclui a lista de siglas e abreviaturas
%\incluilistadesiglas

% Inclui a lista de símbolos
%\incluilistadesimbolos

% ----
% Início do documento
% ----
\begin{document}

% ----------------------------------------------------------
% ELEMENTOS TEXTUAIS
% ----------------------------------------------------------
\textual

\chapter{Introdução}
\label{chapter:introducao}
% Contextualizar:
O concreto é o material de construção mais utilizado, sendo considerado a segunda substância mais utilizada do mundo, perdendo apenas para água \cite{Gagg2014}, sua utilização se estende até as chamadas \sigla{OAE}{Obra de Arte Especiais}, estruturas cujo objetivo é a transposição de obstáculos, como pontes, avenidas, viadutos e túneis. 
De acordo com o Departamento Nacional de Infraestrutura de Transportes (DNIT) (2006), \textit{apud} Mendes (2009), existem 73.000 quilômetros de rodovias pavimentadas e não pavimentadas do modal rodoviário brasileiro, contendo dentro dessas, aproximadamente 5.600 pontes. 
Em 2011 um relatório do Tribunal de Contas da União (Relatório TC 003.134/2011-3) apontou em uma auditoria que o valor estimado das OAE's são da ordem 13 bilhões de reais e esses estão distribuídos em cerca de 4.500 pontes e viadutos na malha federal. Já em 2015 o DNIT contabilizou somente sob sua responsabilidade um total de 5.114 OAE’s.

Para estudar tais estruturas de concreto, em específico suas condições referentes à durabilidade e ampliação de vida útil, foi desenvolvido um campo chamado de \sigla{SHM}{Structural Health Monitoring} \cite{de2011durabiliidade}.
O SHM utiliza principalmente de técnicas computacionais que tem como objetivo avaliar o comportamento real da estrutura e avaliar a sua qualidade como produto de engenharia. 
Essas técnicas tem como objetivo principal informar o estado real em que a estrutura se encontra \cite{inaudi2009structural}.


\section{Problema}

Aplicações do SHM dentro do Brasil são especificadas em normas, como a norma NBR 15575, que é utilizada para verificação de durabilidade para sistemas estruturais de edificações habitacionais.
Normas como essa são necessárias pois por melhor que seja o concreto, atualmente ainda é comum que se manifestem patologias, como fissuras (rachaduras e buracos) ou exposição da sustentação metálica da estruturas (armadura da estrutura), sejam causadas por ações naturais ou por erro humano em sua criação \cite{afonso2021}. 
Dessa forma é de suma importância que ocorra um monitoramento constante das estruturas no geral, até mesmo das que são consideradas saudáveis \cite{statera}.

Por via de regra espera-se que empresas tenham uma equipe específica responsável pelo monitoramento constante de tais estruturas \cite{statera}.
Essa responsabilidade é de alto custo operacional, logo, é sugerido substituir o custo humano por técnicas computacionais que constam no SHM \cite{Liu2002}.

Dentre essas tecnologias, destacam-se a utilização de sensores e heurísticas aplicadas em visão computacional. 
Todavia, essas tecnologias possuem certas desvantagens que desencorajam sua utilização, como custo exorbitante de instalação e manutenção para os sensores e inconsistência de resultado para heurísticas \cite{Zhuang2022}.

Assim sendo, é sugerido utilizar técnicas de aprendizagem profunda que vem obtendo cada vez melhores resultados.
Segundo \citeonline{jain2000statistical} já há bons resultados em diversas áreas como: mineração de dados (\textit{data mining}); análise de imagens; análise de texto; inspeção visual para automação industrial; busca e classificação em base de dados multimídia; reconhecimento biométrico, incluindo faces, íris ou impressões digitais.

\section{Objetivos}
O objetivo deste trabalho é utilizar técnicas de aprendizagem profundo em união com visão computacional para o reconhecimento de padrões de manifestação patológicas a partir de imagens da estrutura.
Para tal, utiliza-se de redes neurais convolucionais aplicadas em análise de imagens para reconhecer padrões de falhas em sistemas estruturais e classificar estes em estruturas saudáveis ou que apresentem  patologias.

\section{Objetivos específicos}
Os objetivos específicos são:

\begin{itemize}
    \item Explorar os problemas que ocorrem em OAE's, como fissuras e exposição de armadura;
    \item Levantamento das bases de dados de imagens de fissuras;
    \item Levantamento dos modelos de aprendizado profundo que melhor se encaixam ao problema;
    \item Validação dos modelos nas bases de dados de imagens de fissuras em OAE's;
    \item Análise da correlação entre as bases de dados de imagens;
    \item Análise dos resultados.
\end{itemize}


\section{Metodologia}

O método utilizado consiste em utilizar quatro bases de dados indicadas por \citeonline{zoubir2021crack} e um subconjunto da união destas para o treinamento de duas arquiteturas de redes neurais convolucionais: \sigla{VGG }{\textit{Visual Geometry Group}} 16 \cite{simonyan2014very} e ResNet \cite{he2016deep}.
Cada arquitetura pode ser utilizada com e sem transferência de aprendizado.

O processo para utilização dessas bases de dados pode ser resumido em duas etapas principais. 
Primeiro, foi feita a busca e a compreensão dos artigos citados por \citeonline{zoubir2021crack}, o que permitiu encontrar as bases de dados online e fazer o \textit{download} delas localmente. 
Em seguida, cada base foi organizada em duas pastas, correspondentes a cada rótulo utilizado. 
Como os autores já disponibilizaram as bases de dados com a rotulação e o pré-processamento necessários, não foi necessário realizar esses procedimentos novamente.

Para a aplicação dos modelos de redes neurais convolucionais, cada uma dessas bases é dividida em duas partes: 90\% para o treinamento e 10\% para os testes. 
Durante o treinamento, a técnica de validação estratificada é utilizada com 10 grupos de divisão, aplicando os 90\% da base de treinamento. 
A acurácia e o \textit{loss} são calculadas para cada grupo, e em seguida, a média, a variância e o desvio padrão são determinados a partir desses valores.
Em seguida, os resultados da validação cruzada são analisados para avaliar a integridade, confiabilidade e robustez dos modelos.

Caso essas métricas sejam satisfatórias, o melhor modelo de cada arquitetura é selecionado para uma bateria de testes. 
Os primeiros testes são realizados com os 10\% mencionados anteriormente, ou seja, os 10\%. 
Por fim, o modelo selecionado é aplicado a todas as outras bases de dados para avaliar a correlação entre elas e determinar qual é a melhor base para treinar um modelo.
A partir desses testes, métricas como acurácia, precisão, sensibilidade, especificidade e $F_{1}$-Score são calculados e utilizados para analise dos modelos e das bases.


\section{Contribuições para a área de pesquisa}

No geral, os resultados que utilizaram modelos de redes neurais convolucionais com transferência de aprendizado obtiveram desempenhos satisfatórios, com acurácias acima de 90\% e valores de $F_{1}-Score$ acima de 80\% para maioria dos casos.
Isso contando os resultados da validação cruzada estratificada e os testes realizados sobre a mesma base em que o modelo foi treinado.

Sobre esses resultados, os modelos VGG16 com transferência de aprendizagem, Densenet e Resnet apresentaram valores médios de $F_{1}-Score$ de 92,93\%, 91,11\% e 91,12\%, respectivamente.
Logo, o modelo VGG16 teve os melhores resultados.

Posteriormente, foram realizadas inspeções visuais de forma minuciosa das bases de dados para argumentar sobre os resultados encontrados.
Em seguida, experimentos foram realizados para avaliar o desempenho de modelos treinados em uma base de dados e testados em outras. 
Os resultados complementam  as análises visuais e mostraram que o uso de um subconjunto que abrange as imagens de todas as bases de dados é a base mais eficaz para treinamento. 
Nesses experimentos, o modelo Resnet foi o que obteve melhor desempenho com $F_{1}-Score$ de 96,44\%, indicando maior capacidade de abstração e generalização de dados em comparação com os outros modelos testados. 


Em suma, os resultados obtidos nesses experimentos são de grande relevância para a escolha de modelos e métodos de treinamento em redes convolucionais. 
Ademais, graças a esses experimentos, foi possível identificar qual base de dados apresentou os melhores resultados quando aplicada em todas as imagens. 
Essas informações são valiosas para orientar futuros trabalhos na área de processamento de imagens e aprendizado de máquina, contribuindo para o aprimoramento de técnicas e algoritmos utilizados em aplicações práticas.
Além disso, com o uso dessas técnicas avançadas de processamento de imagens e aprendizado de máquina, é possível reduzir significativamente o tempo e os custos envolvidos na inspeção e manutenção dessas estruturas, garantindo a segurança e a integridade das mesmas. 
Portanto, esses resultados têm implicações diretas na engenharia civil, auxiliando na manutenção e preservação de infraestruturas.

\section{Organização do texto}

O texto é organizado da seguinte maneira: Este primeiro capítulo apresenta a introdução do projeto. 
No \autoref{chapter:concreto} são apresentados os conceitos de estruturas de concreto e patologia, e sua problemática. 
No \autoref{chapter:deep_learn} é explicado sobre aprendizagem de máquina, desde seu conjunto maior até chegar em redes neurais convolucionais que é o foco desta monografia.

O \autoref{chapter:correlatos} apresenta trabalhos recentes na literatura que empregam redes neurais de forma correlata ao tema desta monografia.
No \autoref{chapter:metodologia} é dissertado a metodologia de como os experimentos que geram os resultados finais foram realizados.
No \autoref{chapter:resultados} são apresentados os resultados.
Por fim, o \autoref{chapter:conclusao} apresenta as conclusões da pesquisa.

\chapter{Concreto, Fissuras e OAE`s}
\label{chapter:concreto}
\section{Considerações iniciais}

Segundo o relatório técnico \citeonline{MineralCommodity2007} o concreto é atualmente considerado o material estrutural mais utilizado no mundo, e ficando em segundo lugar em materiais gerais perdendo apenas para a água. 
Os cálculos apresentados por \citeonline{Gagg2014} relatam que há o dobro de concreto sendo utilizado em construções do que há somando a utilização de aço, alumínio, madeira e plástico, com isso podendo ser encontrados na maioria das categorias de construções, desde casas de alvenaria até pontes, usinas e plataformas de extração petrolíferas \cite{Lima2014}.

Nessa linha de raciocino, é necessário entender o básico de seus conceitos e sua contextualização na problemática desta pesquisa, logo, nesse capítulo serão abordados os conceitos de concreto, sua aplicação em OAE`s, suas variações de aplicações, o limite de sua vida útil e patologias.

\section{Conceitos gerais}

Embora seja um material muito conhecido, ainda há uma falta de entendimento principalmente pelos cidadães comuns, que podem não saber diferencial concreto de cimento já que ambos termos serem comumente usados de forma conjunta \cite{Gagg2014}, 
isso acontece por conta que o concreto é uma massa resultante da mistura de diversas materiais, sendo o principal destes o cimento, que é o material aglutinante deste compósito, agrupando todos os materiais que compõem o concreto. \cite{allen2019fundamentals}.

Essa mistura pode ser modificada, porém a receita padrão é a mistura de água, cimento, agregados miúdos como areia, e agregados graúdo como cascalho, aditivos e adições; dependendo da porcentagem presente de cada ingrediente diferentes características poderão aparecer no concreto, além de impactar sua resistência, que é a quantidade de força de compressão resistida pelo concreto, sendo geralmente medido em \sigla{MPa}{Megapascal} \cite{pinheiro2007fundamentos}. 
Tais diferenças de medidas são desejáveis e que permitem o concreto ser utilizado mundialmente já que cada lugar há diferentes necessidades, locais com muita umidade, muito sol, chuvas fortes ou construções de grande peso como pontes, para cada um desses há uma receita diferente e geralmente cabe ao engenheiro tecnologista do concreto estipular tais características garantindo assim especificações mínimas conforma as normas de projeto de cada país. \cite{izharcomparison}.

A principal característica do concreto é sua versatilidade que acontece por se tratar de uma mistura que em primeiro momento está em um estado físico líquido, podendo ser moldado de acordo com a necessidade alterando sua forma para se adequar a diferentes tipos de estruturas, como pilares, paredes, piso, entre outros \cite{Gagg2014}. Após um período de espera o concreto se enrijece saindo de seu estado líquido até um estado sólido onde se obtêm uma grande resistência à compressão.


\section{Concreto armado}

O concreto armado é um material que emprega o concreto tradicional e uma adição de barras de aço, onde ambos se complementam para resistir respectivamente às forças de compressão e tração \cite{Lima2014}. 
Essas barras de aço, chamadas de armadura do concreto oferecem ao concreto a resistência necessária para combater esforços de tração, necessária em todas as peças estruturais que formam edificações, pontes e outras estruturas. \cite{pinheiro2010estruturas}.

\section{Obra de arte especial}

De acordo com a definição pelo engenheiro civil e pesquisador \citeonline{Ciro2014}, \sigla{OAE}{Obra de Arte Especial} são construções estruturais com finalidade transpor grandes obstáculos, tais quais como rios, desníveis, porções urbanas, entre outros; dessa forma, se configura como OAE estruturas como pontes quando construídas sobre níveis de água, como viadutos quando sobre avenidas ou espaços secos ou como túneis em casos abaixo da superfície.

No Brasil, o principal órgão regulador dessas estruturas, a nível federal, é o \sigla{DNIT}{Departamento Nacional de Infraestrutura de Transporte} autarquia responsável por implementar a política de infraestrutura de transportes terrestres e aquaviários,
estabelecendo as regras de construção e monitoramento que devem ser seguidas pelos construtores \cite{dnitdados}.

Como dito anteriormente, o concreto, principalmente o concreto armado, possui uma incrível resistência á compressão, fazendo-o uma ótima escolha como elemento principal na construção das OAE's, ainda mais quando comparado seu custo operacional e disponibilidade no mercado \cite{santos2008armaccao}. 
Porém como todo material estrutural o concreto possui uma vida útil que é pré-estabelecida em função do seu tipo de aplicação. O material ao longo dessa vida útil sofre com a ação das forças da natureza e depreciação por uso, perdendo parte desta resistência mecânica \cite{santos2008armaccao}. Portanto estudar meios para planejar reparos na estrutura são técnicas importantes no estudo das estruturas de concreto.

Por conta de tais fatores o DNIT torna obrigatório o monitoramento das OAE's,  habitualmente a cada dois anos, sendo obrigatório a presença de inspetores qualificados com anos de experiência e inspetores auxiliares em um processo demorado da criação de um relatório que envolve a observação de toda a OAE, captura de evidências visuais como fotos ou vídeos e descrição por escrito da situação da OAE, de suas falhas e expectativa para os próximos anos \cite{dnit2004}.

\section{Falhas estruturais}

De acordo com o conceito relatado por \citeonline{afonso2021}, falhas estruturais são quando um componente estrutural ou até mesmo toda a estrutura perde a capacidade de uso determinada em projeto, tais falhas podendo ser classificadas de acordo com seu fator expositivo. 
Em especial para esta pesquisa, as falhas causadas por conta de fraturas são chamadas de falha frágil sendo geralmente resultante de se danos acumulados, que com o tempo fazem com que o material estrutural perca sua resistência, dessa forma também sendo configurado como um dano progressivo \cite{anneLink2016}.

 Segundo \citeonline{cremonini1988incidencia}, o concreto é de fato um material com uma resistência muito alta, tanto em questão de cargas quanto em agressões ambientais, porém essa resistência pode vir a ruir, comprometendo sua capacitância de suportar os empenhos solicitados. 
 Ao fazer o estudo dos causadores dessa perda de resistência, a engenharia emprega o termo patologia aos tipos de causas e origens de tais problemas \cite{cremonini1988incidencia}
 
 Á vista disso, é imprescindível  que se tenha a realização de estudos sobre tais patologias e como evitá-las. Com isso, algumas das patologias mais comuns \cite{statera} encontradas são:

\begin{description}
    \item[Fissuras:]
    A ocorrência de fissuras em estruturas que utilizam concreto armado podem implicar em sérios problemas estruturais e simbolizam sérios problemas de estado de conservação, nos piores casos, com a falta de manutenção apropriada faz com que toda a estrutura se comprometa, perdendo todo seu caráter estético, social econômico ademas dos riscos de segurança ao usuário \cite{santos2014patologia}.
    
    Dessa forma, é fato que fissura é um problema de enorme importância que abrange diversas áreas, desde econômicas até da satisfação psicológica dos usuários da estrutura \cite{andrade1998durabilidade}, por conta disso é necessário entender suas causas e origens para descobrir então suas consequências e as remediações necessárias de modo a ter certeza que uma vez reparada, não aconteça da estruturas voltar a se deteriorar \cite{de1998patologia}.
    
    Existem variados tipos de fissuras que podem acontecer em estruturas de concreto armado, sendo que cada fator implica em diferentes tipos de fissuras \cite{nakamura2007}. 
    A falta de umidade no concreto provoca o aparecimento de várias aberturas lineares, esse tipo de fissura é chamado de fissura por retração. 
    Quando é adicionado mais carga do que o calculado em uma estrutura pode acontecer as chamadas fissuras de sobrecarga que costumam ser graves e requerem manutenção urgente. 
    Em estruturas com mais de um material estrutural como a mistura de concreto armado e alvenaria, pode acontecer de os materiais não agirem do mesmo modo frente à ação de dilatação, fazendo acontecer a chamada fissuras higro-térmicas \cite{nakamura2007}.
    
    O DNIT apresenta uma série de regras e de instruções para a realização da manutenção de cada patologia e cada OAE.
    
    Exemplos dessa patologia podem ser vistas em \autoref{fig:fissuras}
    
    \begin{figure}[htb]
    \centering
        \begin{subfigure}{.5\textwidth}
          \centering
          \includegraphics[width=.8\linewidth]{images/mapa_da_obra_img_fissura.jpg}
          % \caption{A subfigure}
          \label{fig:fissura01}
          \fdireta{fig:fissura01}
        \end{subfigure}%
        \begin{subfigure}{.5\textwidth}
          \centering
          \includegraphics[width=.8\linewidth]{images/tudo_construcao_fissura02.jpg}
          % \caption{A subfigure}
          \label{fig:fissura02}
          \fdireta{fig:fissura02}
        \end{subfigure}
    
    \caption{Exemplos de fissuras em estruturas de concreto.}
    \label{fig:fissuras}
    \end{figure}

    
    \item[Corrosão de armadura:]
    
    O concreto armado consegue ser um ótimo material estrutural por conta que uma material consegue remediar os problemas de seus componentes, por exemplo, o concreto tem uma forte alcalinidade, protegendo a armadura contra corrosão \cite{pinheiro2010estruturas}.
    Porém, uma possível patologia, que geralmente acontece pela falha durante a concretagem, ou por erros de \WANDER{cálculo} para determinar a espessura do concreto, podem resultar na exposição da armadura, fazendo com que a exposição ao dióxido de carbono do ambiente faça a corrosão acontecer \cite{statera}. 
    No caso de certas OAE`s, como pontes, existem uma maior quantidade de agentes agressivos, como em zonas marítimas onde há o vento que carrega partículas de sal\cite{statera}.

    Além de ser a patologia mais usual nas estruturas de concreto armado, a corrosão de armadura também é uma das mais perigosos, sendo considerado uma anomalia grave cujo o eventual produto pode vir a ser o colapso total da estrutura\cite{tecnosil_2017}. 
    Posterior ao inicio dessa patologia, se não houver tratamento e manutenção, a corrosão manifesta uma progressão constante e ininterrupta em todos os casos \cite{tecnosil_2017}.
    
    \item[Falhas de Concretagem:]
    
    As falhas de concretagem denominam as irregularidades que ocorrem no período do processo de concretagem por conta de erros de colocação ou compactação do concreto \cite{statera}.
    Segundo SILVA(2011), projetos deficientes, de execução mal feitas são os principais responsáveis pelo surgimento de patologias.
    
    Para reprimir que tais erros ocorram, há uma série de diretrizes que cada projeto deve executar, no caso de OAE`s o órgão emissor dessas diretrizes é o DNIT. 
    Contudo, segundo \citeonline{tecnhe_2010}, no contexto atual da construção civil, não estão sendo mais seguidos os prazos sugeridos para cada etapa construtiva, diminuir o tempo de cada etapa resulta também em menos tempo para planejar e projetar as características da obra e de seus componentes.

    
\end{description}


\section{Problemas/prejuízos causados pelas fissuras}

% Repetitivo?
Embora não apresente problemas sérios diretamente, as fissuras representam o sinal inicial de que um problema sério pode vir a acontecer \cite{alani2014integrated}, por conta disso o DNIT exige que aconteça regularmente uma inspeção \cite{dnit2004}.
Uma inspeção realizada da forma correta consegue produzir várias informações do estado atual da estrutura através da
medição de grandezas físicas, como acelerações, deformações, e/ou deslocamentos.

O problema que o método tradicional de inspeção, que é o manual, pode ser considerado um trabalho extensivo, custoso, e possivelmente perigoso, além de necessitar da contratação técnicos capacitados e de confiança \cite{adhikari2014image}.
Outro grande custo é o carecimento de certas ferramentas e equipamentos, em especial em lugares de difícil acesso onde há a utilização de certos veículos especializados, como caminhão com cesta elevatório \cite{dorafshan2018bridge}.
Há ainda o fator do descontentamento gerado ao publico, pois geralmente há a necessidade de no mínimo limitar o acesso à OAE, o que também causa custos de forma indireta \cite{catbas2018vision}.

\section{Enfoque computacional para solucionar o problema}

Para solucionar tal problema, diversos autores buscam desenvolver novas ferramentas e técnicas para auxiliar o serviço de manutenção e monitoramento. 
Tais autores concordam que as melhores soluções possíveis estão dentro do campo computacional e portanto se faz necessário o estudo com enfoque computacional dentro da engenharia.

\subsection{Sensoriamento}

Uma das primeiras soluções encontradas foi a de utilizar cadeias de sensores que extrairiam dados da ponte em tempo real, para que o técnico responsável tivesse acesso a tais informações sempre que requerido \cite{spencer2019advances}.
Embora para época tenha sido uma alternativa recomendável, mesmo com as evoluções tecnologias, até hoje esta é uma solução problemática por conta dos custos exorbitantes \cite{Zhuang2022}.
Entre os desafios enfrentados se destacam: realizar a instalação da fiação de cabos por toda a estrutura, transmissão e geração de energia para sustentar a cadeia de sensores, transferência de dados entre si e para um sistema receptor, calibração regular dos sensores \cite{catbas2018vision}.

% Apresentando apenas os problemas
% Aprofundar sobre os sensores?

\subsection{Visão computacional}

Com o passar do tempo, as técnicas baseadas em visão computacional ganharam maior autoridade por conta de seus resultados em identificação, classificação e localização de objetos dentro de imagens, por conta disso \citeonline{webb2015categories} apresenta a visão como uma possibilidade para ser utilizada na engenharia civil, em especifico no âmbito da engenharia civil, e do campo de inspeções e monitoramento.

Ao utilizar imagens e vídeos como dados de analise para o computador é possível obter as informações necessárias para o monitoramento \cite{spencer2019advances} sema a necessidade do contato direto humano, evidentemente, sendo essa sua maior vantagem \cite{catbas2018vision}.

Uma outra vantagem da visão computacional é que a estratégia para captura de tais imagens e vídeos não é fixa ou precisa de um equipamento específico, imagens de alta qualidade podem ser registradas até mesmo de câmeras de telefones, mas caso se tenha um maior investimento é possível utilizar câmeras de alta performance podendo ser fixas ou móveis, entretanto, uma tecnologia que vem conquistando espaço são os \sigla{UASs}{Unmanned Aerial Systems}, mais conhecido como drones \cite{dorafshan2018bridge}. Atualmente, os drones tem conquistado espaço em vários ambientes profissionais, sendo considerado uma ferramenta eficiente, contribuitiva e uma alta relação de custo-benefício \cite{zoubir2021crack}. 
Dessa forma sendo uma ótima alternativa dentro do campo das inspeções e monitorações. 

Com essas tecnologias de visão computacional citadas e um investimento mínimo, é possível até mesmo implementar um sistema automatizado de coleta das imagens. Entretanto, mesmo com a automatização da captura de imagens, seja através de câmeras estacionarias, drones ou outro método, ainda há a parte mais tediosa e custosa em questão de tempo, que é a de um técnico especialista realizar a analise interpretativa dos dados coletado \cite{zoubir2021crack}, essa etapa porém pode ser também automatizada a partir de métodos computacionais.

\subsection{Utilização de heuristicas}

Durante o começo das pesquisas de visão computacional, as principais ferramentas a serem utilizadas para detectar problemas em estruturas eram caracterizadas como heurísticas, normalmente esses métodos eram executados utilizando com a aplicação de limiar ou com a criação de filtros específicos \cite{spencer2019advances}, que posteriormente poderiam até ser calculados utilizando aprendizagem de máquina.

Os tipos de filtros mais comuns utilizados que é utilizado até em trabalhos mais recentes são os detectores de bordas, onde a a partir de alguma premissa especificada pelo usuário um algoritmo busca em uma imagem uma série de pontos onde existam mudanças bruscas que interligados configure algum objeto, dessa forma realizando uma segmentação da imagem \cite{ziou1998edge}.

Em um dos primeiros trabalhos específicos da área, \citeonline{abdel2003analysis} utiliza os detectores de borda \textit{Sobel} e \textit{Canny} e as equações transformadas de \textit{Fourier} e \textit{Fast Haar} como gradiente para detecção de borda em 50 fotos adquiridas a partir de câmeras fotográficas e conseguindo uma acurácia dividida entre 86 e 64\% dentre seus métodos, sendo a transformada de \textit{Fast Haar} a que gerou o melhor resultado, porém sendo ainda necessário a interpretação de um usuário para separar o que realmente são fissuras dos erros causados pela textura. 

Já o autor \citeonline{jahanshahi2009survey} compara o método de detecção de borda com a união de certas técnicas morfológicas:

O processo de detecção de bordas é detalhado sobre os cálculos utilizados, sendo utilizados cálculos de derivadas de altovalores e altovetores, gradientes, filtro Gaussiano e Laplaciano, transformadas de \textit{Fast Haar}, limiarização baseada em mediana e uma pequena camada de rede neural de uma camada para caracterizar os objetos gerados, com isso obtendo uma acurácia de 71.5\%, que o próprio autor compara com último trabalho citado, de \cite{abdel2003analysis}, pela similaridade, e explica que a diferença de resultado acontece principalmente pela mudança da direção da luz conforme o tempo do dia. 

O outro método, consiste em tratar a imagem a partir de sua morfologia matemática, ou seja, uma matriz de valores da escala de cinza, os processos morfológicos mais básicos são a dilatação e erosão, a partir da combinação sendo possível realizar os processos de abertura, fechamento, gradiente morfológico, operações \textit{bottom-hat} e \textit{top-hat}, o problema é que esses processos morfológicos necessitam um elemento estruturante de tamanho apropriado para cada tipo de imagem, o que o torna menos dinâmico. Embora o autor não informe a porcentagem de acurácia, ele define esse tipo de abordagem mais efetivo que o detector de bordas pois gera menos ruido que tal.

Para finalizar, o autor \citeonline{dorafshan2019benchmarking} compara a eficiência, precisão e tempo computacional de quatro diferentes filtros aplicados junto à detecção de bordas nos domínios espacial e dois filtros aplicados no domínio de frequência. Como base de dados foi utilizado 50 imagens de concreto saudável e 50 imagens de concreto com presença de fissuras, ambos coletados através de drones. 

O método aplicado consiste em inicialmente converter a imagem para níveis de cinza dando um peso muito maior à cor verde, um valor médio ao vermelho e baixo ao azul. % Coloco a porcentagem?
A partir disso, nessa nova imagem é aplicado o filtro desejado: \textit{Roberts}, \textit{Prewitt}, \textit{Sobel} e \textit{Laplacian of Gaussian} para o domínio espacial e \textit{Butterworth} e \textit{Gaussian} para o dominio de frequência, cada filtro consiste em uma pequena matriz de valores que são utilizados a partir da operação de convolução.
Para o domínio da frequência, é necessário realizar uma transformação do domínio espacial para frequência, que é feito através da transformada de \textit{Fast Fourier}. Após a aplicação de cada filtro, é realizado um aprimoramento na imagem resultante, principalmente normalização a partir da média e desvio padrão, e correção de claridade para aumentar a visibilidade das bordas. Para finalizar é realizado a segmentação da imagem, onde resultará em uma nova imagem com o formato binário onde será positivo os locais que apresenta as rachaduras, essa conversão é realizada com um corte a partir de um limiar encontrado pelo método de \textit{Otsu}.

Para avaliar os resultados, um inspetor revisou as imagens para verificar as imagens e classifica-las de forma a ter 100 \% de acurácia em suas rotulações, logo, foi comparado os resultados dos filtros com o resultado apresentado pelo inspetor, tendo resultado entre 77\% à 92\% de acurácia, 82\% à 88\% de precisão e variação de tempo entre 1.18 e 1.92 segundos, sendo que \textit{Laplacian of Gaussian} foi o filtro com os melhores resultados, com 92\% de acurácia, 88\% de precisão e 1.18 segundos.

\wander{LUCAS EU GOSTEI DESSA PARTE SUA DAS HEURTISTICAS MAS EU RETIRARIA O TITULO DESSA SEÇÃO COMO HEURISTICAS...EU COMPREENDO HEURISTICAS DE OUTRA FORMA...NORMALMENTE AS HEURISTICAS SÃO OS PROBLEMAS DE OTIMIZAÇÃO IGUAL VC FEZ NA SUA IC...ENTÃO VEJA COM A PROFA. NUBIA MAIS EU RETIRARIA SOMENTE ESTE TÍTULO. AI LOGICAMENTE AQUELA PARTE INICIAL QUE VC DEFINIFE DE HEURISTICAS SAI.}

\wander{AQUI EU ACHO QUE NÃO PRECISA DESSA SEÇÃO...ACHO QUE VOCÊ JÁ FAZ O ESTUDO DELA NO ITEM 3 MESMO ENTÃO NÃO PRECISA COLOCAR.}
\subsection{Deep learning}



\chapter{Aprendizado de Máquina}
\label{chapter:deep_learn}
\section{Considerações iniciais}

Deep learning, ou aprendizado reforçado em português, é um ramo do aprendizado de máquina que por sua vez é uma das áreas de estudo nas ciências de inteligencia artificial. Segundo \citeonline{Simon2013od} o aprendizado de máquina tem um foco maior em algoritmos de computação que possuem a capacidade de aprenderem e se aprimorarem, sem a necessidade de uma programação expressa.
A partir disso, o deep learning tenta alcançar esse ato de educar-se a partir de repetidas iterações dentro de uma ambiente correspondente à um objetivo em particular, dessa forma encontrando um resultado para tal objetivo através de reflexões de seus erros em novas tentativas.

Nesse capítulo busca-se introduzir os conhecimentos e aplicações do deep learning:

LEMBRAR DE FALAR DE FUNÇÃO DE ATIVAÇÃO
POOLING
SOFTMAX
DROPOUT
Adam optimizer

\section{Machine learning}

O aprendizado de máquina, do inglês machine learning, faz parte da área de estudo da inteligencia artificial, em especifico as aplicações que buscam emitir um ótimo resultado com base em alguma entrada, sendo que esse resultado é fruto de um processamento realizado por maquinas de forma que há um aprendizado onde se busca uma melhora desse resultado até que se encontre tal resultado ótimo, porém o fator mais importante de tal tecnologia é que tal aprendizagem seja feito apenas pela máquina, no máximo com um auxilio ou supervisão humana \citeonline{ed2021}.

Um dos principais exemplos da aplicação de aprendizado de máquina são os motores de busca da internet como Google, que a partir de buscas anteriores, consegue descobrir quais sites são mais relevantes e com isso recomenda-los mais. 
Outro grande exemplo são os gerenciadores de e-mails eletrônicos como Gmail ou Outlook, que a partir de analise da estrutura dos e-mails e de seu conteúdo categoriza os e-mails recebido como spam, lixo, promoção ou social. 
Á vista disso, pode se definir o machine learning como algoritmos que buscam ter uma maior acurácia em suas predições \cite{ed2021}.

Para ambos os exemplos apresentados, existem diversos caminhos dentro do machine learning que o programador pode seguir para implementar o machine learning e obter uma boa acurácia. Atualmente pode se dividir tais caminhos em quatro diferentes métodos: Aprendizagem supervisionado, aprendizagem não supervisionado, aprendizagem semi-supervisionado e aprendizagem por reforço \cite{Russell2009}.

O aprendizado supervisionado se refere aos casos onde há um programador ou cientista de dados que ativamente que fornece o algoritmo de machine learning com dados de treino já rotulados e definem quais serão as variáveis esperadas de resultado, dessa forma o algoritmo já tem desde o início sua entrada e saída pré-estabelecida e precisa aprender a partir disso \cite{Russell2009}. 
Exemplos desse método são algoritmos de regressão e classificação.

No aprendizado não supervisionado acontece o inverso do primeiro tipo, o algoritmo não tem nenhum auxilio, não há nenhum rotulo em suas entrada e não é especificado que tipo de saída é esperado, com isso o algoritmo se vê obrigado a aprender a relacionar os dados e como eles fazer parte de uma correlação, nessa abordagem pode se dizer que o algoritmo é mais livre para pensar de diferentes maneiras, porém geralmente isso vem ao custo de uma maior custo computacional e de tempo, além de uma maior incerteza que o resultado terá uma boa acurácia \cite{ed2021}. 
Exemplos desse tipo são clustering e grande parte de deep learning, incluindo redes neurais.

Já no aprendizado semi-supervisionado, há uma misturas de ambos aprendizados anteriores, onde se busca deixar o algoritmo com a maior liberdade possível, mas com os dados de entrada rotulados \cite{ed2021}.
Esse é o tipo mais incomum e alguns dos seu exemplos são algoritmos de tradução e de rotulação.

Para finalizar, o aprendizado por reforço é usado para que a máquina aprenda a partir de respostas dadas pelo cientista de dados, dessa forma o algoritmo elabora um resultado e o cientista de dados responde o algoritmo se aquele resultado foi bom ou não, a partir de diversas iterações o algoritmo irá analisar quais métodos de pensamento geralmente levam a um bom resultado e então começará a usa-los de forma mais predominante \cite{Van_Otterlo2012}.
Exemplos desse tipo são robôs autômatos para execução de tarefas, como jogar um jogo ou gerenciamento de recursos.

A maior desvantagem do aprendizagem de máquina é que, pelo menos atualmente, é impossível preparar um algoritmo para qualquer evento, e que mesmo um algoritmo seja ótimo ou melhor que isso é muito improvável que chegue até uma acurácia de 100\%, e um grande exemplo disso é a aplicação de aprendizagem de maquina para implementação de algoritmos capazes de dirigir um carro no transito, pois não há como prever todas as eventualidades possíveis do transito, e qualquer acidente, por menor que seja pode ser fatal \cite{ed2021}.


\section{Redes neurais artificiais}

Assim como já introduzido previamente, as redes neurais artificiais, ou apenas redes neurais, são um método de aprendizado de máquina que tem inspiração no aprendizado de seres, embora tal área seja dona de muitos mistérios, como por exemplo, não se sabe ao certo como acontece o aprendizado, porém há como deduzir técnicas que levam ao ato de aprender, como ler um livro ou aprender através de erros. Tais técnicas ainda não funcionam bem em máquinas, então na verdade o que as redes neurais herdam dos seres pensantes é a arquitetura do órgão responsável pelo pensamento, ou seja o cérebro.

Sabe-se que o cérebro é composto por células chamadas neurônios que se estendem por todo o corpo através do sistema nervoso, os neurônios por si só não são capazes de muita coisa, porém quando existem vários são capazes de compartilhar informações através das chamadas sinapses. 
A partir de observações feitas notou-se que após adquirir conhecimento, as sinapses acontecem de forma diferente, com mais força ou peso, dessa forma se deduz que o ato de adquirir conhecimento acontece com a modificação dos pesos das sinapses \cite{Haykin1998}.

A partir desses conhecimentos, para simular um conjunto de neurônios se implementa uma rede neural, onde há uma função das entradas cujo resultado são propagados para os neurônios seguintes que possui outra função e assim por seguinte nos chamados neurônios intermediários até chegar nos neurônios de resposta que são os finais.
Entre um neurônio e outro existe um certo peso que irá de algum modo modificar o valor ou a informação durante a passagem de tal dado de um neurônio para outros, logo, se espera que modificar tais pesos irá fazer com que a rede neural aprenda e retorne uma resposta conclusiva \cite{lecun2015deep}. Para tal é realizado diversos períodos ou épocas de treinamento para que se possa calibrar o peso ideal, durante tais épocas os pesos devem ser alterado com muita cautela e utilizando fundamentos da matemática.

Espera-se que após um número suficiente de épocas de treinamento, a rede neural se torne proficiente em tal tarefa, com uma precisão aceitável sendo capaz de realizar acertos com dados fora da base de treinamento \cite{Aggarwal2018}.

\subsection{Feed-Forward Propagation}

Propagação para a frente, ou Feed-Forward Propagation, é o tipo de propagação padrão no descobrimento dos parâmetros certos para o processo de aprendizagem de uma rede neural, nesse modelo, o dado de entrada sempre segue em direção à ultima camada \cite{vikashraj2019}, 
ou seja o dados são inseridos na camada inicial e a partir do treino realizado são calculados novo parâmetros da rede neural, após os cálculos da camada inicial os dados seguem para a próxima camada intermediaria onde acontece o mesmo processo de calculo de parâmetros, esse procedimento acontece até que aconteça dos cálculos da camada final.

Segundo \citeonline{Zell2007} o Feed-Forward Propagation é qualquer configuração de rede neural que não há conexões entre neurônios que resultem em um circuito, ou ciclo, dessa forma podendo ser considerado o tipo mais simples de rede neural, onde a informação percorre a rede neural em um sentido apenas.

O processo que acontece no Feed-Forward Propagation pode ser resumido em duas etapas \cite{vikashraj2019}, onde ambas ocorrem em cada neurônio nas camadas intermediarias e na final:

A primeira etapa é a de pré-ativação, onde há uma soma ponderada das entradas, a partir disso se constrói uma transformação linear dos possíveis neurônios. Baseado no resultado desse calculo e na função de ativação os neurônios decidem caso o dado deve ser propagado ou não.

A segunda etapa é de ativação, que acontece quando o calculo da soma ponderada é passada para a função de ativação. A função de ativação é uma função matemática que incorpora tipos não lineares à rede neural. Os tipos mais comuns de função de ativação usados são as funções sigmoide, hiperbólico, tangente, \sigla{ReLU}{retificador} e Softmax.

\subsection{Backward propagation}

Propagação para trás, ou Backward Propagation é um dos principais aspectos do treino de redes neurais, sendo o algoritmo de aprimoramento dos pesos de uma rede neural com base na taxa de erro atingida pela última iteração, ou época. Esse aprimoramento permite o decremento da taxa de erro, fazendo com que sua precisão aumente geralmente \cite{Anas2019}.

Segundo \citeonline{brilliant_back}, a implementação do backward propagation é feita através do gradiente descendente aplicada em uma função de erro, seu algoritmo calcula o gradiente da função de erro de acordo com os pesos da rede neural seguindo uma ordem inversa, com a camada final sendo a primeira à ser calculada e a camada inicial por último, sendo que para cada camada subsequente são utilizados os gradientes das camadas antecedentes.



\section{Deep learning}

De acordo com \citeonline{lecun2015deep}, o deep learning permite que modelos computacionais aprendam, a partir de varias camadas de processamento, à interpretar variados níveis de abstração.
Tais métodos conseguiram evoluir de forma considerável o estado da arte em diversas áreas, como detecção de imagens e compreensão de linguagem natural, sendo que a tecnologia mais atual. 

A principal característica  do deep learning e que o faz se integrar à classe de aprendizado não supervisionado é que não é necessário a realização de um pré-processamento, normalmente esses algoritmos conseguem utilizar como entradas dados sem uma estrutura fixa, como imagens e textos \cite{ibm2020}.
O algoritmo de deep learning é capaz de abstrair diferentes características de suas entradas, a partir disso são realizados diferentes iterações onde será processado o quanto aquela característica, ou sua ausência, se faz importante para obter uma acurácia alta.


O deep learning tem como inspiração para seu aprendizado de máquina, o que faz nós humanos aprender, ou pensar, que são os neurônios e como eles se relacionam uns com os outros, por conseguinte a sua estrutura é a chamada rede neural \cite{lecun2015deep}, onde existem várias camadas de processamento e para cada uma existem um certo número de neurônios.
Cada neurônio se comunica com todos os neurônios da próxima camada, sendo que a primeira camada é a camada de entrada no algoritmo e a última camada é a camada de saída, e todo o processamento é realizado nas camadas intermediarias. 
Seguindo essa lógica, cada comunicação se estabelece através de um calculo e um peso, com isso o que o deep learning faz para aprender, é mudar esses pesos pra mudar o resultado final. 

Durante o processamento do deep learning o algoritmo realiza variados testes, buscando a arquitetura de rede neural ideal, modificando o número de camadas intermediarias, a quantidade de neurônios por camadas e o peso entre eles, porém é bem comum que os próprios supervisores do algoritmo realizem testes com estruturas pré definidas para acelerar o processo e estabelecer que tipo de processamento cada camada irá representar ou que tipo de característica será explorada \cite{Bengio2013}.

Algumas das principais utilizações do deep learning são feitas através de dois modelos de algoritmos, \sigla{CNN}{Convolutional Neural Network} e \sigla{RNN}{Recurrent neural network}. As CNN`s são utilizadas principalmente em visão computacional e classificação de imagens, o que será melhora abordado no próximo capitulo. Já as RNN`s são redes neurais que formam grafos que representam uma sequencia temporal, fazendo com que se externe a dinamicidade da ação do tempo sobre os dados \cite{Abiodun2018}.

\section{CNN}

Explicar CNN

\subsection{Arquitetura AlexNet}

% Contextualização do modelo AlexNEt
AlexNet  \cite{alexnetAnalyticsVidhya2021} é uma arquitetura premiada pela competição \textit{ImageNet} de 2012, popularizando a utilização de camadas de convolução cada vez mais profundas, uma de suas principais características e que o tornava um modelo com ótimas performances as custa de um considerável aumento no custo computacional \cite{krizhevsky2017imagenet}. 
O modelo conta com oito camadas com cerca de 63 milhões de parâmetros com capacidade de aprendizado, separados em cinco camadas de convolução e três camadas totalmente conectadas onde são utilizados \textit{pooling} e \textit{dropout}, nas saídas de suas camadas é utilizado a função de ativação Relu, exceto na saída final, onde se utiliza da função de ativação \textit{Softmax} \cite{alexnetAnalyticsVidhya2021}.

% Trabalho Correlato que utiliza AlexNEt
Em seu trabalho, \citeonline{kim2018automated} vê a necessidade da aplicação de métodos de \textit{Deep Learning} aplicados na detecção de fissuras em imagens de superfícies de estruturas, segundo ele a escolha se dá principalmente por conta da capacidade do \textit{Deep Learning} em superar dificuldades encontradas em experimentos ao ar livre, como a mudança de iluminação.
Para alcançar tal objetivo, \citeonline{kim2018automated} busca inicialmente a criação de um banco de dados, tal é utilizado a ferramenta \textit{ScrapeBox} para realizar uma busca pela internet e coletar imagens a partir de uma palavra-chave, após essa coleta se formou uma base de dados de cerca de 7,000 imagens, que após um pré-processamento e divisão se tornou 50,000 imagens, sendo separado 10,000 imagens em 5 classes: fissura, juntas (Poucas), juntas (Muitas), superfície saudável e presença de plantas. Isso ocorre pois o autor descreve que ter apenas dois classificadores (Com fissura e Sem fissura), resultam em muitos falsos positivos quando aplicados em imagens com muitos obstáculos, fazendo a acurácia subir de 32\% para 98\%.

Como modelo computacional, foi escolhido a arquitetura AlexNet pelo fato de ser uma arquitetura bem difundida e que apresenta bons resultados em aplicações semelhantes. 
Para o modelo de treinamento do modelo AlexNet, optou-se por realizar apenas o aperfeiçoamento com base nas novas classes a partir do modelo pré-treinado no banco de dados \textit{imagenet}, segundo o autor essa decisão leva em conta que treinar o modelo do início toma o tempo de uma semana mesmo em um computador de alto padrão, o que impossibilita realizar testagens rápidas com diferentes configurações. O modelo AlexNet utilizado consiste em cinco camadas de convolução seguidas por camadas de \textit{max-pooling} e três camadas totalmente conectadas com uma saída \textit{softmax} que corresponde a possibilidade de cada classe, além de possuir a função ReLU para diminuir o efeito de gradiente em seus neurônios.

Como último estagio de processamento, é desenvolvido um método que cria um especie de janela deslizante de tamanha bem menor que a imagem original, essa janela irá percorrer a imagem original e aplicar o segmento da janela como entrada no modelo, dessa forma obtendo probabilidades para cada pedaço da imagem para que possa ter um calculo de correlação entre as porcentagens e diminuir o erro, além de detectar a posição das rachaduras com maior exatidão.

Durante o treino, houve a separação da base de dados em 80\% para treino e 20\% para teste e ocorreu em 60 épocas, o que levou 316 minutos, lembrando que isso sobre o modelo pré-treinado. O resultado foi que já na época 8 houve uma acurácia de 98\% e 99\% na época 22 que se manteve nesse faixa até o final da validação.

Para avaliar o modelo, o autor foi até ambientes de construção e capturou imagens utilizando câmera do celular e drones para aplicar o modelo já totalmente treinado e compara-lo com o resultado feito manualmente pelo autor, vale notar que essas imagens contém diversos obstáculos como canos, moldes, entre outros. O resultado foi a presença de uma acurácia acima de 90\% para todos os testes, junto com uma média de 86\% de precisão.

O autor considera o resultado excelente, porém aponta suas limitações em distinguir fissuras de objetos que são indistinguíveis apenas com visão e conclui o estudo com um teste em campo utilizando um drone e realizando uma detecção em tempo real que apresenta uma precisão de 88\% a nível de pixel e detectando 15 de 16 fissuras, embora ele explique que essa fissura não detectada realmente é bem fina e que a imagem capturada por drone é levemente borrada e atrapalhe na detecção do modelo.

\subsection{Arquitetura DenseNet}

% Contextualização do modelo DenseNet
Redes convolucionais densamente conectadas, ou arquitetura DenseNet \cite{huang2017densely} é um modelo recente que promete ser o próximo passo no que diz respeito em aumentar a profundidade das redes convolucionais \cite{PabloDenseNet2018}. 
Após a popularização do aumento da profundidade das camadas convolucionais por conta do modelo AlexNet \cite{krizhevsky2017imagenet} em 2012, houve um constante aumento na profundidade de diversos modelos chegando em números massivos, porém esse aumento no caminho que a informação precisa percorrer desde a camada inicial até a camada final se tornou tão grande que o dado pode ser deturpado e não se tornar nada \cite{PabloDenseNet2018}, além do custo computacional exorbitante. 
Para contornar tais problemas, o DensetNet busca simplificar o modelo de conectividade entre as camadas enquanto garanta que o fluxo de informações não seja perturbado, dessa forma foi utilizado um modelo de reutilização dos extratores de características, realizando conexões que ligam todas camadas diretamente com todas as outras \cite{huang2017densely}. Por conta disso o modelo DenseNet precisa de menos parâmetros e elimina a necessidade de camadas redundantes. Se comparado com classificador de uma rede neural genérica que depende dos resultados da última camada de rede, o DenseNet consegue utilizar de forma mais inteligente os resultados já produzidos, obtendo um resultado melhor em performance e precisão.

% Trabalho Correlato que utiliza DenseNet
Os estudos propostos por \citeonline{qiao2021computer} implicam que há necessidade de evoluir os métodos atuais de monitoramento e manutenção de pontes, em especifico a detecção de fissuras em sua estrutura de concreto, sendo ferramentas como sensores demasiadamente caras, como alternativa é sugerido a utilização de visão computacional. 
Ao comparar as possíveis ferramentas que podem ser aplicadas com visão computacional, se percebe que opções como processamento de imagem baseado em percolação, limiarização e detectores de bordas são escolhas populares, porém que falham quando se buscam uma maior automatização tendo em vista o ambiente efêmero em que são aplicadas.

Ao buscar uma solução, o autor realiza uma busca dentro da área do \textit{Deep Learning}, explicando que essa área possui três características desejadas: Robustez, capacidade de aprendizado e automatização. 
Robustez se refere a capacidade do modelo em extrair as características desejadas de uma imagem com estabilidade, capacidade de aprendizado, permitindo que o modelo aprenda a identificar as características que importantes para o problema específico e automatização diz respeito à capacidade do modelo conseguir operar sem ou quase sem auxilio humano.

O modelo escolhido por \citeonline{qiao2021computer} é o DenseNet \cite{huang2017densely}, por conta de suas inovações no campo do \textit{Deep Learning}, e ainda fundamenta mencionando que a principal característica do DenseNet, que são as camadas totalmente conectadas utilizam menos largura de banda e diminuem a sobrecarga de armazenamento, todavia, que a principal vantagem do DenseNet é o decréscimo do custo computacional por camada na rede neural causado pela reaproveitamento de recursos dentro da rede, dessa forma permitindo que o DenseNet precise re-aprender menos características, consequentemente diminuindo consideravelmente a quantidade de parâmetros e cálculos, além de possuir uma ótima performance em fugir do \textit{overfitting}.

Ainda não satisfeito, o autor argumenta que combinar \textit{Deep Learning} com outros algoritmos de processamento de imagem pode produzir melhores resultados, à vista disso sendo implementado o modulo EMA (\textit{Expectation-Maximization Attention}) baseado no algoritmo de maximização de expectativa (EM) \cite{li2019expectation}, aprimorando o resultado fazendo com que o DenseNet tenha mais atenção nas áreas mais danificadas. 
Esse algoritmo é aplicado durante as últimas camadas de \textit{polling}, transformando a arquitetura em EMA-DenseNet

Como base de dados, o autor utilizou a base de dados de \citeonline{yang2018automatic} que contém cerca de 800 imagens para validação, entretanto o autor não achou satisfatório realizar o treinamento a partir dessas imagens por conta de considerar inconsistente com os ambientes reais, dessa forma, o próprio autor fotografou pontes da região de Xuzhou, na China, resultando em 1800 imagens de fissuras e  2500 imagens com exposição de armadura.
Para finalizar, as imagens foram recortadas em menores imagens, giradas e aprimoradas, além de denotar todas as fissuras a nível de pixel por escolha do autor de utilizar aprendizado supervisionado.

O treinamento foi realizado através do algoritmo ADAM de otimização \cite{kingma2014adam} que otimiza certas variáveis de configuração gerando uma maior convergência de resultados. O método de avaliação escolhido foi de verificar através de quatro cálculos: 
PA, que representa a proporção do número de pixels preditos sobre o total de pixels, 
MPA utiliza de PA para calcular a proporção de pixels em cada classe que estão corretamente classificados para gerar uma média entre todas as classes, 
MIoU é o calculo da média da proporção de \textit{cross over} em cada classe, 
e por último o cálculo de precisão que é a porcentagem dos pixels corretamente classificados em relação à todos os pixels.

Para demonstrar a capacidade do EMA-DenseNet, também são treinados as arquiteturas FCN \cite{yang2018automatic}, SegNet \cite{badrinarayanan2017segnet}, DeepLab v3+ \cite{chen2018encoder} e SDDNet \cite{choi2019sddnet}, para comparação. O treinamento  foi feito durante 20.000 iterações, onde já na iteração 2.000 já uma convergência muita rápida com MIoU chegando à um valor estável de 87.42\%. Segundo o autor o processo de treinamento já prova que o algoritmo é confiável e o modelo após as 20.000 iterações apresentam um MIoU, PA, MPA e precisão de 87.42\%, 97.58\%, 92.59\% e 81.97\% respectivamente. Com o resultado de todos os modelos é exibido que de todos os modelos, o EMA-DenseNet teve os melhores resultados em quase todos os quesitos, exceto em PA, onde o modelo FCN obteve 97.96\%.

Como forma de testar os resultados obtidos, os próprios autores coletam uma base de dados com bem mais ruídos, e danos muito maiores, nesse caso foi apresentado um MIoU, PA, MPA e precisão de 79.87\%, 97.31\%, 86.35\% e 74.70\% respectivamente, e tendo as melhores resultados em comparado com os outros modelos por uma grande diferença em maioria.

\subsection{Arquitetura VGG}
% Contextualização do modelo VGG16
o grupo de pesquisa VGG (\textit{Visual Geometry Group}) \cite{simonyan2014very}, foi vencedor na modalidade de classificação com localização no desafio \textit{ImageNet} de 2014 e tendo uma boa colocação na de apenas classificação, onde alcançou 92,7\% no teste de acurácia entre a base de dados do \textit{ImageNet}, constituída por 14 milhões de imagens divididas em 1000 classes \cite{ILSVRC15}. 
A principal ideia proposta pelo grupo é a de usar filtros de convoluções muito menores, matrizes de tamanho 3x3 com o deslocamento de apenas um pixel por toda a rede neural, sendo bem menor que outros modelos padrões da época como AlexNet \cite{krizhevsky2017imagenet}, que utiliza matriz de convolução de tamanho 11x11 com deslocamento de 4 pixels.
O funcionamento de tal ideia é de utilizar mais camadas para compensar o menor tamanho do filtro, contudo, já que no final de cada camada há uma função de ativação, há uma maior utilização de tal processo, fazendo com que a rede se torne mais descriminante e resultando também no aumento da convergência da rede.
No que diz respeito ao custo computacional, mesmo com aumento na quantidade de camadas, a matriz 3x3 requer menos parâmetros no geral, logo, requer menor potência computacional\cite{GreatLearningVgg16}.

A configuração dos modelos propostos é de utilizar blocos constituídos de um à três camadas de convolução com filtro de tamanho 3x3, seguidos por uma camada de \textit{max-pooling} de tamanho 2x2, a quantidade de blocos é variada durante o artigo original e finalizado utilizando camadas densas de convolução \cite{simonyan2014very}, porém a configuração que obteve o melhor resultado no desafio \textit{ImageNet} foi a de profundidade 16, deixando tal configuração conhecida como VGG16. Em específico, a VGG16 é constituída por dois blocos iniciais, cada um com duas camadas de convolução seguida pela camada de \textit{max-pooling}, em seguida três blocos intermediários, cada um com três camadas de convolução seguida pela camada de \textit{max-pooling} e para finalizar três camadas densas de convolução.

% Trabalho Correlato que utiliza VGG16

O artigo de \citeonline{gopalakrishnan2018crack} apresenta uma diferente técnica dos trabalho citados anteriormente aplicada em conjunto com a arquitetura VGG16. Segundo o autor, aprimorar as ferramentas dos engenheiros os auxiliará de forma a garantir melhores condições de realizar as auditorias sobre as infraestruturas civis, além de diminuir os custos de realizar tais processos frequentemente.
Como solução, a principal ferramenta apresentada é a utilização de drones, que vem sendo cada vez mais populares em diversos ramos, desde aplicações em monitoramento pós desastres naturais até inspeções de construções, tendo um grande impacto por deixar tais atividades mais fáceis, seguras e com um ótimo custo-beneficio \cite{vidyadharan2017civil}.

Nos anos recentes, até mesmo drones mais populares tem a capacidade de alta movimentação, sinal com grande espaço de cobertura e câmeras capazes de capturar imagens em alta definição com resolução suficiente para capturar até mesmo micro-fissuras sem precisar estar muito próximo da estrutura \cite{gopalakrishnan2018crack}. 
Só com as imagens capturadas com drone já é possível para um engenheiro ter um laudo sobre as patologias que uma estrutura contém, servindo como uma ótima ferramenta, contudo, o autor vai além e propões a automatização desse processo aplicando algoritmos de \textit{deep learning}, em especifico utilizar redes neurais convolucionais por conta de diversos experimentos recentes que mostram sua grande efetividade em processar imagens e vídeos, ademais de usar como entrada dados sem processamento ou sem demarcação de dados.

Uma grande dificuldade enfrentada por \citeonline{gopalakrishnan2018crack} é a falta de uma base robusta de imagens de fissuras estruturais, fazendo com que ele opta-se por um método mais barato e eficiente, que é a de utilizar modelos computacionais que já foram treinados em bases de dados massivas, como o \textit{ImageNet} e apenas realizar um aprimoramento em cima dos parâmetros desse modelo a partir das novas classes buscadas, no caso duas classes são utilizadas: com ou sem fissura. 
Esse método foi possível utilizando o \textit{framework} Keras \cite{chollet2015keras} que realiza a implementação do modelo VGG16 com a opção de utilizar os parâmetros pré treinados com base no \textit{ImageNet}.

Como base de dados utilizado, o autor realiza a captura de 130 imagens de estruturas de concreto, onde 80 destas apresenta fissuras.
Essa captura é feita a partir de um drone de alta performance com uma câmera de alta definição, as imagens obtidas podem variar de acordo com a proximidade do drone à superfície e angulo da câmera no momento da captura, atribuindo mais flexibilidade para realizar várias iterações em uma mesma fissura \cite{gopalakrishnan2018crack}.

Graças a rede VGG16 já estar pré treinada e já saber extrair características da imagem, o autor treina apenas o último classificador para as classes desejadas. 
Diferentes classificadores são treinados também na base do \textit{ImageNet} e testados para ser usado em conjunto ao VGG16, tais modelos de classificadores são: Rede neural de uma camada e 256 neurônicos com otimizador ADAM \cite{kingma2014adam}, floresta aleatória de 300 árvores, floresta extremamente aleatória de 300 árvores, máquina de vetores de suporte do tipo linear, e regressão logística.
Para finalizar essa camada é utilizado a função de ativação \textit{softmax} para gerar o resultado final.

A base de dados foi dividida em 70\% treino, 10\% validação e 20\% teste, e com treino realizado ao longo de 50 iterações, o resultados dos testes foram análisados em acurácia, precisão, \textit{recall}, \textit{F1-score} e \textit{Cohen`s Kappa score}.
Com toda a fase de treinamento e validação realizada, os testes mostram que o modelo VGG16 utilizando transferencia de aprendizado e o classificador final com uma rede neural de uma camada de 256 neurônio ou com regressão logistica apresentaram ótimos resultados, com a acurácia de 0.89, precisão de 0.91, \textit{recall} de 0.89, \textit{F1-score} de 0.89 e \textit{Cohen`s Kappa score} de 0.788.

\subsection{Arquitetura Inception V3}
% Contextualização do modelo Inception V3

Inception v3 \cite{szegedy2015going} é modelo de localização, reconhecimento e classificação de imagem que atualmente alcança uma acurácia superior a 78,1\% na base de imagens do \textit{ImageNet}. A arquitetura procura aumentar a profundidade sem aumentar o número de parâmetros da rede para que não aconteça o estouro de gradiente, por conta disso o modelo conta com menos de 25 milhões de parâmetros separados em 42 camandas, que pode ser comparado com os 60 milhões em 8 camadas do AlexNet \cite{alexnetAnalyticsVidhya2021}.

% Trabalho Correlato que utiliza Inception V3

O artigo de \citeonline{zoubir2021crack}, apresenta uma comparação interessante entre modelos de redes neurais convolucionais.

\chapter{Metodologia}
\label{chapter:metodologia}
\section{Considerações iniciais}
Neste capítulo apresenta a metodologia utilizada para execução deste trabalho, desde escolha de base de dados, montagem e escolha das arquiteturas de CNN`s, até o método de treinamento e métricas para avaliação.

\section{Base de dados}
A partir da recomendação de \citeonline{zoubir2021crack}, foram selecionados as bases de \citeonline{zhang_base2018}; \citeonline{maguire2018sdnet2018}, chamada de SDNET2018; \citeonline{zoubir2021crack} e \citeonline{xu2019automatic}.

\subsection{Obtenção e especificidades das bases}
\label{sub:bases}

As bases utilizadas foram encontradas e baixadas da Internet, a partir de seus artigos. 
Todas as bases são rotuladas para classificação, com imagens divididas em dois rótulos. 
Embora as bases usem diferentes nomes para os rótulos, elas podem ser renomeadas e reorganizadas como "positivo" para imagens com fissuras e "negativo" para imagens sem fissuras sem que altere o significado.

A base elaborada por \citeonline{zhang_base2018} é composta por um total de 40.000 imagens de resolução $227 \times 227$, as quais foram divididas em 20.000 positivas e 20.000 negativas.
As imagens foram tiradas em construções do campus \sigla{METU}{Middle East Technical University} utilizando um \textit{smartphone}, obtendo 458 imagens de alta resolução ($4032 \times 3024$).
Fora a divisão e rotulação, não acontece  nenhum processamento de \textit{data augmentation}.
Por fim, as imagens apresentam diferentes condições de iluminação e texturas.

SDNET2018 é um conjunto de dados de imagens rotuladas de uso para algoritmos de detecção de fissuras em concreto baseados em inteligência artificial \cite{maguire2018sdnet2018}.
Este conjunto contém mais de 56.000 imagens de decks de pontes, paredes e pavimentos de concreto com e sem fissuras, com fissuras tão estreitas quanto 0,06 mm e tão largas quanto 25 mm. 
Além disso, o conjunto de dados inclui imagens com várias obstruções, como sombras, aspereza superficial, escalonamento, bordas, buracos e detritos de fundo.

Seus criadores, \citeonline{maguire2018sdnet2018}, capturaram 230 imagens de superfícies de concreto trincadas e não trincadas (54 decks de pontes, 72 paredes, 104 pavimentos) usando uma câmera digital Nikon de 16 MP. 
As superfícies foram localizadas no sistema de laboratório \sigla{SMASH}{\textit{Utah State University system, material, and structural health}} e nas estradas e calçadas do campus da universidade. 
Cada imagem foi segmentada em sub-imagens de $256 \times 256$ pixels.

O artigo de \citeonline{zoubir2021crack} é o que recomenda as outras bases de dados, no entanto, a metodologia empregada consiste em utilizar uma base de dados própria do autor.
Para tal, foram coletadas 572 imagens de resolução $5152 \times 3864$ de plataformas e pilares de concreto com fissuras, sendo que para melhorar a diversidade do conjunto de dados proposto as imagens apresentam diferentes condições de iluminação e superfície como: Rugosidade, cor, umidade e luz forte. 
As imagens são então recortadas em 1304 imagens fissuradas e 5634 não fissuradas.
Porém, apenas 1.050 imagens fissuradas e 2.772 não fissuradas são disponibilizadas para \textit{download}.

Diferentes tipos e tamanhos de fissuras estão presentes nas imagens. 
Além disso, o conjunto de dados apresenta alterações de superfície desafiadoras, tais como manchas e marcas. 
É importante mencionar que algumas imagens que não possuem fissuras contêm juntas de concreto, que podem ser confundidas com fissuras durante a classificação, e ainda podem ser encontrados pequenos defeitos.

Já o artigo de \citeonline{xu2019automatic}, utiliza a base de dados proposta por \citeonline{LiLiangBase} aplicando \textit{data augmentation}.
A base original é composta por 2.068 imagens de imagens de fissuras, coletadas pela câmera \textit{Phantom 4 Pro's CMOS} com resolução de $1024 \times 1024$.

Por conta da base original possuir apenas imagens com fissuras, \citeonline{xu2019automatic} relata que foi necessário recortar as imagens originais em quatro imagens de $512 \times 512$, assim obtendo 8.272 imagens.
Em seguida, foi necessário remover as imagens que não apresentavam condições adequadas para o processamento por estarem borradas. 
Por fim, é utilizado a operação corte central aleatório para reduzir as imagens para a resolução $224 \times 224$ e escolhido imagens de forma aleatória para rotacionar.

Assim, resultando em 6.069 imagens, sendo 4058 imagens fissuradas e 2011 imagens não fissuradas.
Entretanto, ao fazer o \textit{download} da base e organizá-la pelas instruções do autor, há uma incoerência por conta de haver 6.070 imagens, sendo essas 4.056 imagens com fissuras e 2.014 imagens sem fissuras.
Porém, como no arquivo baixado consta as classes de todas as imagens, é considerado como uma adição posterior ao artigo.

A \autoref{tab:bases_dados} apresenta algumas das características principais sobre as bases citadas.
Sendo essas características a quantidade de imagens, sua divisão por rótulo e sua resolução em pixels.


\begin{table}[htb]
\centering
\caption{Bases de dados}
\label{tab:bases_dados}
\begin{tabularx}{\textwidth}{l|c|c|c|c} \hline
Base de dados & Fissuradas & Saudáveis & Total & Dimensões \\ \hline
\citeonline{zhang_base2018}         & 20.000    & 20.000    & 40.000 & $227 \times 227$ \\
SDNET2018   & 8.484   & 47.608     & 56.092 & $256 \times 256$ \\
\citeonline{zoubir2021crack}        &  1.050    & 2.772     & 3.822 & $200 \times 200$ \\
\citeonline{xu2019automatic}        & 4.056     & 2.014     & 6.070 & $224 \times 224$ \\ \hline
Total:                              & 33.590    & 72.394   & 105.984 & - \\ \hline
\end{tabularx}
\fdadospesquisa
\end{table}

\subsection{União das bases}
\label{subchap:uniao}

Acredita-se que, além da realização de experimentos individuais para cada base de dados, conduzir um experimento adicional com a fusão dessas bases poderá gerar resultados mais consistentes, devido à maior diversidade de características combinadas.

Entretanto, apenas com uma breve análise da \autoref{tab:bases_dados}, é possível perceber alguns problemas no escopo da união dessas bases. 
Primeiro que cada base apresenta imagens de dimensões diferentes sendo que uma rede neural convolucional consegue aceitar como entrada apenas imagens com uma dimensão fixa. 
Outro problema é o desbalanceamento de classes que ocorre já que há mais que o dobro de imagens saudáveis do que há de fissuradas.

\section{Pré-processamento}

Os processamentos prévios realizados são justamente para solucionar os problemas para a união das bases, citados anteriormente.
Sendo respectivamente: Redimensionar essas imagens para uma dimensão em comum \cite{thakur2022}, e selecionar apenas uma amostra balanceada da base de dados, já que nesse caso há um alto desbalanceamento

Por conta disso, os experimentos adicionais com a fusão das bases comentada na \autoref{subchap:uniao} será feito com um subconjunto de todas as 105.984 imagens.
Para isso, utilizaremos um subconjunto aleatório de 40.000 imagens selecionadas a partir do total de 105.984 imagens. 
Esse subconjunto será referido como 'Subconjunto 40k' e será composto por 20.000 imagens pertencentes à classe "Fissuradas" e 20.000 imagens pertencentes à classe "Saudáveis".
Logo, esta monografia apresenta um total de cinco bases de dados.

Dando continuidade aos processamentos realizados, durante a leitura dos lotes de imagens, ocorre a normalização dos valores, que envolve a transformação dos valores das três camadas das imagens, que normalmente variam de 0 a 255, em valores de ponto flutuante entre 0 e 1.
Além disso, por conta das redes neurais precisarem de um tamanho de entrada fixo, todas as imagens foram reduzidas para o tamanho $200 \times 200$ pixels durante a leitura dos lotes.
Fora esses, não há necessidade de realizar outros pré-processamentos nas bases uma vez que as imagens já foram tratadas, recortadas e rotuladas \cite{great_preprocess2022}.

Caso fosse necessário aumentar a quantidade de imagens, vale citar processos de \textit{data augmentation}, como rotação, alteração de brilho, correção de gama, entre outros. 
Porém não se viu necessário por já se ter uma base de dados robusta, com grande quantidade de imagens e que apresenta diferentes características e variações da classe alvo.

\section{Arquiteturas}
As arquiteturas de rede neural convolucional escolhidas para serem testadas são VGG16 \cite{simonyan2014very}, 
% Inception v3 \cite{szegedy2015going}, 
DenseNet \cite{huang2017densely} 
e ResNet \cite{he2016deep}, por conta de seus resultados em trabalhos correlatos, já comentados na \autoref{dl:arquiteturas}. 

A implementação destes se dá na linguagem Python 3 \cite{py3} através das bibliotecas Tensorflow \cite{tensorflow2015-whitepaper} e Keras \cite{chollet2015keras} que oferecem tais modelos já configurados, sendo necessário apenas a instanciação desses modelos e sua configuração inicial.

No site do Keras \cite{chollet2015keras}, é possível encontrar informações sobre os modelos utilizados, incluindo acurácia \textit{Top}-1, acurácia \textit{Top}-5, quantidade de parâmetros, profundidade e custo de inferência na GPU em milissegundos. 
Esses detalhes são apresentados na \autoref{tab:info_arq} para os modelos empregados.

A acurácia \textit{Top}-1 e a acurácia \textit{Top}-5 referem-se à habilidade dos modelos de classificar imagens corretamente em um conjunto de validação pertencente ao conjunto de dados \textit{ImageNet}. 
A acurácia \textit{Top}-1 mede se o modelo previu corretamente a classe mais provável entre todas as possíveis classes, enquanto a acurácia \textit{Top}-5 avalia se o modelo previu corretamente a classe correta entre as cinco mais prováveis.

A profundidade do modelo se refere ao número de camadas que contêm parâmetros. 
O custo de inferência é uma métrica que mede o tempo médio que um modelo de aprendizado de máquina leva para processar uma única amostra de entrada e produzir uma saída correspondente. 
O custo de inferência na GPU é calculado considerando o uso de uma GPU Tesla A100 com um \textit{batch} de tamanho 32.

\begin{table}[htb]
\centering
\caption{Informações das arquiteturas utilizadas}
\label{tab:info_arq}
\begin{tabularx}{\textwidth}{l|c|c|c|c|c} \hline
Modelo & Acurácia \textit{Top}-1 & Acurácia \textit{Top}-5 & Parâmetros & Profundidade & Custo (ms)\\ \hline \hline
VGG16 & 71.3\% & 90.1\% & 138.4 M & 16 & 6.6\\ \hline
DenseNet201 & 77.3\% & 93.6\% & 20.2 M & 402 & 6.7\\ \hline
ResNet152V2 & 78.0\% & 94,2\% & 60.4 M & 307 & 4.2\\ \hline
\end{tabularx}
\fdireta{chollet2015keras}
\end{table}

As arquiteturas VGG, Densenet e Resnet possuem diversas variações.
Entretanto, nesta monografia foram utilizados o VGG16, DenseNet201 e ResNet152V2 em específico por apresentarem as maiores acurácias \textit{Top}-1 e \textit{Top}-5 de suas arquiteturas.


\section{Método}

Cada modelo de rede neural convolucional é treinado e testado em cada base.
E posteriormente avaliada utilizando todas as bases como referência.
O treinamento do modelo é realizado utilizando o método de validação cruzada estratificada \textit{K-fold}.

\subsection{validação cruzada}

A validação cruzada \textit{K-fold} funciona de modo a separar os dados de entrada em $k$ grupos de tamanhos semelhantes, e então executar $k$ iterações onde um desses grupos será escolhido para validar o modelo e o restante dos grupos servirão para o treinamento do modelo, sendo que a cada iteração é selecionado um grupo diferente para validação \cite{kohavi1995study}. 
Já sua versão estratificada \textit{K-fold} distribui os dados de entrada igualando a quantidade das classes entre os grupos, dessa forma os grupos terão uma representação balanceada das classes, evitando vieses na avaliação do modelo \cite{geron2019hands}.

\subsection{Divisão e aplicação das bases de dados}

Para cada modelo, cada base de dados é dividida em 90\% e 10\% de forma a manter o balanceamento de cada classe.
Sendo os 10\% reservados para avaliações, ou seja, para testes.
Já os 90\% são utilizados como entrada para validação cruzada estratificada, utilizando 10 grupos ($k$ = 10).
Por conta dessa divisão, pode se afirmar que 90\% desses 90\% serão utilizados para treinamento e os outros 10\% desses 90\% serão utilizados para validação, embora aconteça um rotacionamento desses dados.


\subsection{Cálculo dos resultados do treino e validação}

Para o treinamento dos modelos, é utilizado a função \textit{fit} da biblioteca Keras.
Essa função permite escolher qual método de acurácia e de \textit{loss} será utilizada para calcular esses valores e retornar seus resultados durante a execução

Antes de utilizar um modelo na validação cruzada estratificada, o estado inicial do modelo, incluindo seus parâmetros e pesos, é guardado. 
Assim, ao iniciar cada grupo, o modelo é reiniciado para esse estado inicial. 
E ao final da execução de cada grupo (treinamento e validação), o modelo é salvo na memória.
Assim, no final da validação cruzada, terá salvo na memória $k$ modelos treinados.
Essa abordagem garante que cada grupo tenha sempre um mesmo início e que informações de um grupo não passem para outros, já que o modelo é revertido para suas condições iniciais a cada novo grupo. 
Dessa forma, o treinamento ou validação de um grupo não afeta o treinamento, ou o desempenho, dos outros grupos.

Os resultados finais dessa etapa se dão em acurácia e \textit{loss} e embora sejam retornados valores de acurácia e \textit{loss} para tento treino quanto validação.
Apenas os valores da validação são utilizados para a analise dos modelos.
A acurácia retornada é o número de acertos sobre a quantidade total de imagens da validação.
Esse conceito é melhor explicado na \autoref{sub:calcTreinTest}, e demonstrada na \autoref{eqn:acuracia}.
Já a para o calculo do \textit{loss}, é utilizado a função de perda \textit{categorical cross entropy}, que é melhor explicado na \autoref{sub:cce}

Por fim, a análise dos resultados da validação cruzada usam o resultado final de cada grupo.
Estes são registrados em um conjunto para calcular a média, variância e desvio padrão do modelo para a base utilizada.
A partir desses dados é possível obter uma estimativa mais confiável do desempenho do modelo em dados não vistos. 
A média representa a estimativa pontual do desempenho do modelo, enquanto o desvio padrão representa a variabilidade dos resultados obtidos nos diferentes conjuntos de teste.
A variação dos resultados obtidos em diferentes conjuntos de teste pode fornecer informações adicionais sobre a robustez do modelo.


\subsection{categorical cross entropy}
\label{sub:cce}
A \textit{cross entropy} é uma medida comumente usada em problemas de classificação, e a \textit{categorical cross entropy} é uma adaptação da \textit{cross entropy} para o caso de múltiplas classes. 
Ela é uma medida de dissimilaridade entre as distribuições de probabilidade, uma calculada pelo modelo e outra sendo a distribuição verdadeira das classes \cite{geeron2017handson}.

A fórmula da \textit{categorical cross entropy} é simples, e pode ser observada em \autoref{eqn:cce}.
Nela, temos $M$ como quantidade de classes; 
$y$ como indicador binário (0 ou 1) que representa se a classe $c$ é a classificação correta para imagem $o$; 
$p$ é a probabilidade prevista da imagem $o$ ser da classe $c$.


\begin{equation}
\label{eqn:cce}
CCE = -\sum_{c=1}^My_{o,c}\log(p_{o,c})
\end{equation}
\fdireta{equacoesML}

Se o problema de classificação envolver apenas duas classes, o \textit{categorical cross entropy} pode ser resumida para o \textit{\textit{binary cross entropy}}.
Que pode ser observado em \autoref{eqn:cce2}, respeitando as mesmas variáveis citadas anteriormente.
\begin{equation}
\label{eqn:cce2}
CCE_{M=2} = -{(y\log(p) + (1 - y)\log(1 - p))}
\end{equation}
\fdireta{equacoesML}


O objetivo da otimização é minimizar a \textit{cross entropy} , ou seja, ajustar os parâmetros do modelo para que as predições fiquem o mais próximo possível das classes verdadeiras. 
Isso é feito utilizando técnicas de otimização como o gradiente descendente \cite{Goodfellow2016}.

\subsection{Cálculos da acurácia de treino e testes}
\label{sub:calcTreinTest}

Para minimizar o custo computacional, caso a validação cruzada produza resultados com baixo desvio padrão, é escolhido o grupo que apresentou o melhor desempenho para realizar os testes, em vez de treinar um novo modelo.
Logo, o melhor modelo irá classificar os 10\% reservados para testes.
Por fim, esse mesmo modelo irá classificar todas as imagens das outras bases e a união dessas outras base.
Esses testes não são aplicados na base 'Subconjunto 40k', já que seria redundante e possivelmente prejudicial já que ela pode conter dados em que esse modelo foi treinado.

Esses experimentos têm  o intuito de verificar se o modelo selecionado tem a capacidade de generalização dos dados.
Ou seja, se realmente consegue classificar uma imagem se apresenta fissura de um modo abrangente, ou se é limitado à base treinada.
Fundamentado nesses resultados, é possível analisar como o modelo se comportaria em experimentos mais diversificados.

Esses testes seguem os métodos de avaliação recomendados em \citeonline{xu2019automatic}, onde os resultados obtidos são classificados nas seguintes categorias:

\begin{itemize}
\item \sigla{TP}{Verdadeiros positivos}: O número de classes positivas previstas corretamente como classes positivas. 
Assim, TP se refere ao número de fissuras que são classificadas corretamente como fissuras.

\item \sigla{TN}{Verdadeiros negativos}: O número de classes negativas previstas corretamente como classes negativas. 
Logo, TN se refere ao número de superfícies saudáveis que são classificadas corretamente como superfícies saudáveis.

\item \sigla{FP}{Falso positivos}: O número de classes negativas previstas incorretamente como classes positivas. 
Portanto, FP se refere ao número de superfícies saudáveis que são incorretamente identificadas como fissuras.

\item \sigla{FN}{Falso negativos}: O número de classes positivas previstas incorretamente como classes negativas. 
Então, FN se refere ao número de fissuras que são incorretamente identificadas como superfícies saudáveis.
\end{itemize}

Com esse dados, é possível obter uma diversificada quantidade de informações \cite{geron2019hands}.
Dentre elas, as utilizadas nesta monografia são:

\begin{itemize}
\item Acurácia: 
Razão entre o número de instâncias classificadas corretamente e o número total de instâncias, representando a efetividade geral do classificador. 
Para este caso, a acurácia se refere à proporção de fissuras e fundos que são classificados corretamente. Sua fórmula de cálculo é mostrada na \autoref{eqn:acuracia}:

\begin{equation}
\label{eqn:acuracia}
Acc = \frac{TP + TN}{TP + TN + FP + FN}
\end{equation}

\item
Precisão: 
É a proporção de instâncias verdadeiramente positivas entre todas as instâncias classificadas como positivas. 
No contexto desta monografia, a precisão se refere à proporção de verdadeiras fissuras em todas as instâncias classificadas como fissuras pelo modelo. 
A fórmula de cálculo é mostrada na \autoref{eqn:precisao}:

\begin{equation}
\label{eqn:precisao}
P = \frac{TP}{TP + FP}
\end{equation}

\item
Sensibilidade (\textit{Recall}): 
De todas as instâncias positivas, a sensibilidade determina qual porcentagem é identificada corretamente, representando a eficácia de um classificador para identificar instâncias positivas. 
A sensibilidade corresponde à proporção de quantas verdadeiras fissuras são classificadas como fissuras. 
A fórmula de cálculo da sensibilidade é mostrada na \autoref{eqn:recall}:

\begin{equation}
\label{eqn:recall}
R = \frac{TP}{TP + FN}
\end{equation}

\item
Especificidade: 
Razão entre o número de instâncias negativas classificadas corretamente e o número total de instâncias negativas, representando a eficácia geral do classificador na identificação de instâncias negativas. 
A especificidade em questão refere-se à proporção de fundos verdadeiros que são classificados como fundos. 
Sua fórmula de cálculo é mostrada na \autoref{eqn:espec}:

\begin{equation}
\label{eqn:espec}
E = \frac{TN}{TN + FP}
\end{equation}


\item
\textit{$F_{1}$-Score}: É uma métrica especialmente útil em casos em que as classes positiva e negativa possuem números desproporcionais de instâncias ou quando a taxa de falsos positivos é considerada mais prejudicial que a taxa de falsos negativos.
Essa medida é calculada a partir da combinação da precisão (P) e do \textit{recall} (R) do modelo, o que permite avaliar tanto a capacidade de um modelo em classificar corretamente as instâncias positivas quanto a sua habilidade em evitar falsos positivos.
Logo, um modelo experimental é comprovadamente mais eficaz quando o $F_{1}$-Score é mais elevado \cite{geron2019hands}.
Sua fórmula de cálculo é mostrada na \autoref{eqn:f1}:

\begin{equation}
\label{eqn:f1}
F1 = \frac{2*P*R}{P+R} = \frac{2*TP}{2*TP+FP+FN}
\end{equation}

\end{itemize}

\chapter{Resultados}
\label{chapter:resultados}
\section{Considerações iniciais}
Neste capítulo é descrito quais configurações iniciais são utilizadas para a realização de todos os experimentos.
Em seguida são apresentados os resultados que cada arquitetura utilizada obteve, junto com a explicação destes.
Para finalizar, os resultados de cada arquitetura são postos lado a lado, para uma comparação direta.

\section{Configurações}
Para realizar os experimentos, primeiramente foi definido uma \textit{seed} de valor 240 para possíveis replicações do resultado.

A implementação dos modelos é provida inteiramente pelas bibliotecas Tensorflow \cite{tensorflow2015-whitepaper} e Keras \cite{chollet2015keras}, dessa forma se mantêm as configurações de camada padrões, comentadas em \autoref{dl:arquiteturas}. 
Exceto as camadas superiores que são refeitas para aceitar um tamanho de imagem diferente do padrão.

Os parâmetros utilizados são declarados durante a compilação, ou inicialização, do modelo.
A compilação do modelo é realizada utilizando o otimizador Adam \cite{kingma2014adam} com taxa de aprendizagem $1e^{-7}$, e como função de ativação das camadas finais (Camadas de classificação), a função \textit{softmax}. 

Além disso, durante a compilação do modelo é definido como respostas esperadas a acurácia (\autoref{eqn:acuracia}) e o \textit{loss}, calculado utilizando o método \textit{categorical cross entropy} (\autoref{eqn:cce}).


Para cada arquitetura, variando entre com ou sem \sigla{TL}{\textit{Transfer learning}}, e para cada base de dados, o treinamento é realizado utilizando a validação cruzada \textit{stratified K-fold} com 10 grupos ($k=10$).
Cada um dos treinamentos de cada grupo das validações cruzadas é feito com 100 épocas.
Esse valor foi escolhido por conta de testes realizados em modelos que não utilizavam de \textit{transfer learning}, onde não era observado nenhum resultado aceitável com menos épocas.
Por conta que há comparação dos modelos, para todos foram utilizados os mesmos parâmetros.
Dessa forma, até mesmo os que utilizam \textit{transfer learning} foram treinados com 100 épocas, o que difere dos trabalhos correlatos (\autoref{chapter:correlatos}), que utilizam bem menos épocas.



\section{Resultados do modelo VGG16}
\subsection{VGG16 sem transferência de aprendizado}

Os resultados dos experimentos com o modelo VGG16 sem transferência de aprendizado são apresentados na \autoref{tab:media_kf_vgg} para a validação cruzada estratificada e \autoref{tab:10_vgg} para os testes realizados nos 10\% de testes de cada base.

\begin{table}[htb]
\centering
\caption{Resultados da validação cruzada estratificada 10-\textit{fold} para o VGG16 sem transferência de aprendizado.}
\caption*{
Considere: 'Acc' como acurácia da validação; 'E' como \textit{loss} da validação; 'Var' como variância; $\sigma$ como desvio padrão.
}
\label{tab:media_kf_vgg}
\begin{tabularx}{\textwidth}{|X|p{2.2cm}|p{2.2cm}|p{2.2cm}|p{2.2cm}|p{2.2cm}|}
\hline
Base de dados & \citeonline{zhang_base2018} & \citeonline{maguire2018sdnet2018} & \citeonline{zoubir2021crack} & \citeonline{xu2019automatic} & Subconjunto 40k \\ \hline \hline
Média (Acc) & 98,90\% & 84,87\% & 72,51\% & 66,82\% & 81,86\%\\ \hline
Var (Acc) & 0,000454\% & 0,000001\% & 0,000209\% & 0,000026\% & 0,002027\%\\ \hline
$\sigma$ (Acc) & 0,21\% & 0,01\% & 0,14\% & 0,05\% & 0,45\%\\ \hline \hline
Média (E) & 3,56\% & 40,74\% & 59,11\% & 62,79\% & 39,76\%\\ \hline
Var (E) & 0,0023\% & 0,0011\% & 0,0017\% & 0,0006\% & 0,0076\%\\ \hline
$\sigma$ (E) & 0,48\% & 0,33\% & 0,41\% & 0,24\% & 0,87\%\\ \hline

\end{tabularx}
\fdadospesquisa
\end{table}

\begin{table}[htb]
\centering
\caption{Resultados do melhor modelo nos testes para o VGG16 sem transferência de aprendizado.}
\label{tab:10_vgg}
\begin{tabularx}{\textwidth}{|X|p{2.2cm}|p{2.2cm}|p{2.2cm}|p{2.2cm}|p{2.2cm}|}
\hline
Base de dados & \citeonline{zhang_base2018} & \citeonline{maguire2018sdnet2018} & \citeonline{zoubir2021crack} & \citeonline{xu2019automatic} & Subconjunto 40k \\ \hline \hline
Acurácia & 99.22\% & 84.88\% & 72.51\% & 66.89\% & 81.62\% \\ \hline
Precisão & 99.65\% & 0.00\% & 0.00\% & 66.89\% & 90.11\% \\ \hline
Sensibilidade & 98.80\% & 0.00\% & 0.00\% & 100.00\% & 71.05\% \\ \hline
Especificidade & 99.65\% & 100.00\% & 100.00\% & 0.00\% & 92.20\% \\ \hline
$F_{1}-Score$ & 99.22\% & 0.00\% & 0.00\% & 80.16\% & 79.45\% \\ \hline
\end{tabularx}
\fdadospesquisa
\end{table}

Ao analisar os resultados da validação cruzada apresentados na \autoref{tab:media_kf_vgg}, podemos afirmar que o modelo é consistente, independentemente da base de dados utilizada. 
Isso é evidenciado pela baixa variância, que nunca excede 0,001\%, e o desvio padrão, que nunca é maior que 1\%.

Em seguida, pode ser observado que em todas as bases de dados, com exceção de \citeonline{zhang_base2018}, há um \textit{loss} muito alto, de 40\% para cima.
Isso demonstra que o modelo treinado nessas bases está com um desempenho ruim e, possivelmente, com problemas graves de \textit{overfitting} ou \textit{underfitting}.
Entretanto, esses mesmos modelos estão com acurácias que não demonstram o mesmo, com valores acima de 60\%.

No caso dos estudos de \citeonline{maguire2018sdnet2018} e \citeonline{zoubir2021crack}, os problemas de desempenho podem ser atribuídos ao desbalanceamento de classes em seus conjuntos de dados. 
Isso é reforçado pela análise da \autoref{tab:10_vgg}, que possui uma precisão de 0\% e especificidade de 100\% que mostra que todas as classificações foram feitas para o rótulo desbalanceado.

Já para o caso de \citeonline{xu2019automatic}, o mesmo problema de desbalanceamento acontece, porém nesse caso o rótulo de maior quantidade é o positivo, ou fissurado.
Isso pode ser percebido pela especificidade de 0\% e sensibilidade de 100\% na \autoref{tab:10_vgg}.

É interessante observar que, mesmo sendo balanceada, a base 'Subconjunto 40k' apresentou um valor de \textit{loss} elevado. 
Nesse caso, ao analisar os demais resultados apresentados na \autoref{tab:10_vgg}, é possível argumentar que o modelo não conseguiu aprender suficientemente as características necessárias para obter um desempenho superior a 90\%, que seria o ideal. 
Apesar disso, é importante ressaltar que o modelo ainda apresentou um resultado final aceitável, com um f1-score de cerca de 80\%. 
É possível que esse resultado tenha sido influenciado por fatores como o número insuficiente de épocas ou um \textit{learning rate} muito baixo, entre outros possíveis motivos.

Por fim, a única exceção, a base de \citeonline{zhang_base2018} apresenta ótimos resultados, todos acima de 98\%.
Porém, seu \textit{loss} de 3,5\% o que demonstra que o modelo ainda tem uma margem para melhorar na minimização da função de perda durante o treinamento.
No geral, as métricas apresentadas na \autoref{tab:10_vgg} indicam que o modelo foi capaz de aprender com sucesso as características relevantes para classificar corretamente a grande maioria dos exemplos em ambas as classes.


\subsection{VGG16 com transferência de aprendizado}

Os resultados dos experimentos com o modelo VGG16 com transferência de aprendizado são apresentados na \autoref{tab:media_kf_vggTL} para a validação cruzada estratificada e \autoref{tab:10_vggTL} para os testes realizados nos 10\% de testes de cada base.

\begin{table}[htb]
\centering

\caption{Resultados da validação cruzada estratificada 10-\textit{fold} para o VGG16 com transferência de aprendizado.}
\caption*{
Considere: 'Acc' como acurácia da validação; 'E' como \textit{loss} da validação; 'Var' como variância; $\sigma$ como desvio padrão.
}
\label{tab:media_kf_vggTL}
\begin{tabularx}{\textwidth}{|X|p{2.2cm}|p{2.2cm}|p{2.2cm}|p{2.2cm}|p{2.2cm}|}
\hline
Base de dados & \citeonline{zhang_base2018} & \citeonline{maguire2018sdnet2018} & \citeonline{zoubir2021crack} & \citeonline{xu2019automatic} & Subconjunto 40k \\ \hline \hline
Média (Acc) & 99,90\% & 93,46\% & 98,74\% & 99,65\% & 94,74\%\\ \hline
Var (Acc) & 0,00002\% & 0,00097\% & 0,00244\% & 0,00070\% & 0,00072\%\\ \hline
$\sigma$ (Acc) & 0,04\% & 0,31\% & 0,49\% & 0,27\% & 0,27\%\\ \hline \hline
Média (E) & 0,41\% & 18,68\% & 4,69\% & 1,34\% & 13,85\%\\ \hline
Var (E) & 0,0005\% & 0,0049\% & 0,0327\% & 0,0064\% & 0,0067\%\\ \hline
$\sigma$ (E) & 0,23\% & 0,70\% & 1,81\% & 0,80\% & 0,82\%\\ \hline
\end{tabularx}
\fdadospesquisa
\end{table}

\begin{table}[htb]
\centering

\caption{Resultados do melhor modelo nos testes para o VGG16 com transferência de aprendizado.}
\label{tab:10_vggTL}
\begin{tabularx}{\textwidth}{|X|p{2.2cm}|p{2.2cm}|p{2.2cm}|p{2.2cm}|p{2.2cm}|}
\hline
Base de dados & \citeonline{zhang_base2018} & \citeonline{maguire2018sdnet2018} & \citeonline{zoubir2021crack} & \citeonline{xu2019automatic} & Subconjunto 40k \\ \hline \hline
Acurácia & 99.90\% & 93.67\% & 97.12\% & 99.84\% & 94.70\% \\ \hline
Precisão & 99.95\% & 87.41\% & 94.44\% & 100.00\% & 97.65\% \\ \hline
Sensibilidade & 99.85\% & 67.92\% & 93.41\% & 99.75\% & 91.60\% \\ \hline
Especificidade & 99.95\% & 98.26\% & 98.28\% & 100.00\% & 97.80\% \\ \hline
$F_{1}-Score$ & 99.90\% & 76.44\% & 93.92\% & 99.88\% & 94.53\% \\ \hline
\end{tabularx}
\fdadospesquisa
\end{table}


É necessário uma análise mínima para reconhecer a diferença entre os resultados obtidos ao utilizar a transferência de aprendizado no VGG 16 e quando essa técnica não é aplicada.

A base de \citeonline{zhang_base2018} sem a transferência de aprendizado obteve bons resultados, mas demonstrava capacidade de melhora por conta de seu \textit{loss}.
Com a transferência de aprendizado, houve um aumento da acurácia e do $F_{1}-Score$, além de reduzir o valor de \textit{loss} para as casa decimais.

A utilização da transferência de aprendizado proporcionou melhorias na capacidade de classificação das bases de dados de \citeonline{maguire2018sdnet2018}; \citeonline{zoubir2021crack} e \citeonline{xu2019automatic}, que anteriormente apresentavam limitações. 
Os resultados obtidos após a aplicação da transferência de aprendizado foram satisfatórios e demonstraram que o modelo aprendeu a classificar as imagens de forma mais eficaz.
Entretanto, em especial a base de \citeonline{maguire2018sdnet2018} obteve resultados abaixo da média ao comparar com as outras bases.
Infelizmente, apenas com os dados presentes nos experimentos não há como definir exatamente o motivo.
Porém, pode se argumentar que o causador deste problema é o desbalanceamento, ou que a base apresenta cenários difíceis para o aprendizado do modelo.

A base de dados 'Subconjunto 40k', que anteriormente não conseguiu atender aos requisitos necessários para alcançar os objetivos desta monografia, foi capaz de alcançá-los através da utilização da transferência de aprendizado. 
No entanto, apesar de ter uma acurácia e $F_{1}$-Score de 94\%, seu valor de \textit{loss} de 13\% indica que ainda há espaço para melhorias. 
Esse resultado pode sugerir que a base de dados seja mais desafiadora, uma vez que é uma combinação de várias outras bases e, portanto, apresenta uma variação maior de características.

É crucial destacar que os valores de variância e desvio padrão continuam extremamente baixos, o que indica a consistência do modelo.
Essa consistência é um indicativo de que o modelo é robusto e pode ser generalizado para outras bases de dados.
No entanto, é importante notar que houve um ligeiro aumento desses valores em comparação com os modelos que não utilizaram transferência de aprendizado. 
Esse aumento pode ser explicado pela incorporação de novas informações do conjunto de dados de origem durante o processo de transferência de aprendizado, o que pode levar a uma maior variabilidade nos resultados. 
Em geral, esse aumento não afeta significativamente a qualidade do modelo, mas deve ser considerado ao avaliar sua performance.

Por fim, ao comparar os resultados do modelo com e sem a transferência de aprendizado, observou-se que os resultados com a técnica foram satisfatórios, ao contrário dos resultados sem. 
Por esse motivo, os experimentos sem transferência de aprendizado foram descontinuados, já que eles dobram o custo computacional sem trazer benefícios significativos.


\section{Resultados do modelo Densenet}

Os resultados dos experimentos com o modelo Densenet com transferência de aprendizado são apresentados na \autoref{tab:media_kf_dense} para a validação cruzada estratificada e \autoref{tab:10_dense} para os testes realizados nos 10\% de testes de cada base.

\begin{table}[htb]
\centering
\caption{Resultados da validação cruzada estratificada 10-\textit{fold} para o Densenet.}
\caption*{
Considere: 'Acc' como acurácia da validação; 'E' como \textit{loss} da validação; 'Var' como variância; $\sigma$ como desvio padrão.
}
\label{tab:media_kf_dense}
\begin{tabularx}{\textwidth}{|X|p{2.2cm}|p{2.2cm}|p{2.2cm}|p{2.2cm}|p{2.2cm}|}
\hline
Base de dados & \citeonline{zhang_base2018} & \citeonline{maguire2018sdnet2018} & \citeonline{zoubir2021crack} & \citeonline{xu2019automatic} & Subconjunto 40k \\ \hline \hline
Média (Acc) & 99,85\% & 91,83\% & 97,77\% & 99,08\% & 93,48\%\\ \hline
Var (Acc) & 0,00004\% & 0,00077\% & 0,00300\% & 0,00172\% & 0,00062\%\\ \hline
$\sigma$ (Acc) & 0,07\% & 0,28\% & 0,55\% & 0,41\% & 0,25\%\\ \hline \hline
Média (E) & 0,68\% & 29,87\% & 6,86\% & 2,70\% & 21,23\%\\ \hline
Var (E) & 0,001\% & 0,013\% & 0,034\% & 0,013\% & 0,009\%\\ \hline
$\sigma$ (E) & 0,31\% & 1,16\% & 1,85\% & 1,15\% & 0,97\%\\ \hline
\end{tabularx}
\fdadospesquisa
\end{table}

\begin{table}[htb]
\centering
\caption{Resultados do melhor modelo nos testes para o Densenet.}
\label{tab:10_dense}
\begin{tabularx}{\textwidth}{|X|p{2.2cm}|p{2.2cm}|p{2.2cm}|p{2.2cm}|p{2.2cm}|}
\hline
Base de dados & \citeonline{zhang_base2018} & \citeonline{maguire2018sdnet2018} & \citeonline{zoubir2021crack} & \citeonline{xu2019automatic} & Subconjunto 40k \\ \hline \hline
Acurácia & 99.92\% & 91.51\% & 97.64\% & 99.51\% & 93.03\% \\ \hline
Precisão & 99.90\% & 80.00\% & 98.98\% & 100.00\% & 95.12\% \\ \hline
Sensibilidade & 99.95\% & 58.49\% & 92.38\% & 99.26\% & 90.70\% \\ \hline
Especificidade & 99.90\% & 97.40\% & 99.64\% & 100.00\% & 95.35\% \\ \hline
$F_{1}-Score$ & 99.93\% & 67.57\% & 95.57\% & 99.63\% & 92.86\% \\ \hline
\end{tabularx}
\fdadospesquisa
\end{table}

O modelo Densenet demonstra ter uma ótima acurácia, com valores de acurácia acima de 90\% e com desvio padrão na casa do decimais.
Entretanto apresenta valores de \textit{loss} relativamente altos, e com variância deste \textit{loss} acima de 1\%, podendo representar que o modelo tem instabilidade.


Adicionalmente, a análise das bases de dados SDNET2018 de \citeonline{maguire2018sdnet2018} e 'Subconjunto 40k' revela que há uma redução significativa na taxa de verdadeiros positivos (Sensibilidade), o que afeta negativamente o valor do $F_{1}-Score$. 
Este resultado sugere que o modelo está ainda classificando muitas imagens como positivas, ou com fissuras, quando na verdade não estão. 
Uma das possíveis razões para isso é que o modelo treinado nessas bases não foi capaz realmente de diferenciar algumas texturas das imagens saudáveis, das imagens que possuem fissura, ou seja, diferenciar totalmente o negativo do positivo.
Esse argumento é reforçado ao considerar que os modelos treinados nessas bases possuem valores de \textit{loss} consideravelmente altos  em comparação com as outras bases, com valores acima de 20\%.

A base de \citeonline{zoubir2021crack} apresenta alta acurácia, com valores acima de 97\%.
Entretanto,m há um considerável diminuição para o $F_{1}-Score$, causada pelo sua sensibilidade menor que 93\%, que em conjunto com seu \textit{loss} de praticamente 7\%, demonstra uma fragilidade do modelo treinado nessa base.

Por fim, as bases de \citeonline{xu2019automatic} e \citeonline{zhang_base2018}, apresentam ótimos resultados, com acurácia e $F_{1}-Score$ bem proximos de 100\%.
Além disso, o baixo valor de \textit{loss} de ambos modelos demonstram que os modelos estão bem ajustados.
Embora no caso de \citeonline{xu2019automatic}, o valor de \textit{loss} ainda permite melhoras ao modelo.

\section{Resultados do modelo ResNet}

Os resultados dos experimentos com o modelo Resnet com transferência de aprendizado são apresentados na \autoref{tab:media_kf_res} para a validação cruzada estratificada e \autoref{tab:10_res} para os testes realizados nos 10\% de testes de cada base.

\begin{table}[htb]
\centering
\caption{Resultados da validação cruzada estratificada 10-\textit{fold} para o Resnet.}
\caption*{
Considere: 'Acc' como acurácia da validação; 'E' como \textit{loss} da validação; 'Var' como variância; $\sigma$ como desvio padrão.
}
\label{tab:media_kf_res}
\begin{tabularx}{\textwidth}{|X|p{2.2cm}|p{2.2cm}|p{2.2cm}|p{2.2cm}|p{2.2cm}|}
\hline
Base de dados & \citeonline{zhang_base2018} & \citeonline{maguire2018sdnet2018} & \citeonline{zoubir2021crack} & \citeonline{xu2019automatic} & Subconjunto 40k \\ \hline \hline
Média (Acc) & 99,77\% & 91,28\% & 96,16\% & 98,51\% & 92,83\%\\ \hline
Var (Acc) & 0,00006\% & 0,00153\% & 0,01029\% & 0,00197\% & 0,00116\%\\ \hline
$\sigma$ (Acc) & 0,08\% & 0,39\% & 1,01\% & 0,44\% & 0,34\%\\ \hline \hline
Média (E) & 1,06\% & 35,76\% & 10,37\% & 4,62\% & 26,21\%\\ \hline
Var (E) & 0,002\% & 0,029\% & 0,077\% & 0,027\% & 0,007\%\\ \hline
$\sigma$ (E) & 0,42\% & 1,70\% & 2,77\% & 1,66\% & 0,81\%\\ \hline
\end{tabularx}
\fdadospesquisa
\end{table}

\begin{table}[htb]
\centering
\caption{Resultados do melhor modelo nos testes para o Resnet.}
\label{tab:10_res}
\begin{tabularx}{\textwidth}{|X|p{2.2cm}|p{2.2cm}|p{2.2cm}|p{2.2cm}|p{2.2cm}|}
\hline
Base de dados & \citeonline{zhang_base2018} & \citeonline{maguire2018sdnet2018} & \citeonline{zoubir2021crack} & \citeonline{xu2019automatic} & Subconjunto 40k \\ \hline \hline
Acurácia & 99,92\% & 91,73\% & 97,38\% & 99,34\% & 92,65\% \\ \hline
Precisão & 99,90\% & 80,48\% & 97,98\% & 100,00\% & 94,99\% \\ \hline
Sensibilidade & 99,95\% & 59,79\% & 92,38\% & 99,01\% & 90,05\% \\ \hline
Especificidade & 99,90\% & 97,42\% & 99,28\% & 100,00\% & 95,25\% \\ \hline
$F_{1}-Score$ & 99,93\% & 68,61\% & 95,10\% & 99,50\% & 92,45\% \\ \hline
\end{tabularx}
\fdadospesquisa
\end{table}

Em geral, o modelo ResNet alcança ótimos resultados, com acurácias e $F_{1}$-Score consistentemente acima de 90\% para a maioria dos casos. 
A única exceção é a base de dados SDNET 2018, proposta por \citeonline{maguire2018sdnet2018}, onde a variância e o desvio padrão da acurácia começam a se tornar significativos, com valores maiores do que 0,01\% e 1\%, respectivamente. 
Além disso, a base de \citeonline{maguire2018sdnet2018} apresenta um grande \textit{loss}, representando uma possível falta de aprendizado por conta do modelo.

Embora o erro nas demais bases seja considerável, já que todos estão acima de 1\%, é importante ressaltar que o $F_{1}-Score$ dessas bases ainda é alto. 
Isso sugere que, mesmo com erros maiores, o modelo ResNet ainda é capaz de realizar uma classificação eficiente. 
No entanto, é importante levar em consideração que, em determinados casos, um erro maior pode ter consequências mais significativas e, portanto, medidas devem ser tomadas para reduzir esse erro.
Para isso, sugere-se realizar ajustes dos hiperparâmetros do modelo, ou mesmo mudar o tipo de pré-processamento de dados para melhorar a qualidade da entrada para o modelo.

\section{Comparação dos modelos e das bases}
\label{sub:compa}

A análise conjunta dos resultados obtidos pelos modelos com transferência de aprendizado VGG16, Densenet e Resnet, permite a sua comparação.
Fazendo o calculo de média aritmética dos resultados de $F_{1}-Score$ dos modelos.
A escolha do $F_{1}-Score$, se deve ao fato de que ele é uma medida que leva em consideração tanto a precisão quanto a revocação do modelo. 
Isso significa que o $F_{1}$-Score é uma métrica mais robusta do que a acurácia em situações em que as classes estão desbalanceadas ou quando o objetivo é minimizar tanto os falsos positivos quanto os falsos negativos.

Com base na análise dos resultados dos testes, é possível verificar que os modelos VGG16, Densenet e Resnet apresentaram valores médios de $F_{1}-Score$ de 92,93\%, 91,11\% e 91,12\%, respectivamente.
Observando esses resultados, é possível concluir que o modelo VGG16 obteve o melhor desempenho médio em comparação aos outros modelos avaliados. 
No entanto, é importante salientar que essa conclusão deve ser interpretada com cuidado e levando em consideração o contexto específico da aplicação dos modelos, bem como as características das bases de dados utilizadas no experimento.

Com relação às bases de dados utilizadas, verificou-se que a base de \citeonline{zhang_base2018} apresentou os melhores resultados em todas as métricas avaliadas. 
É interessante notar que essa base foi a única em que o modelo VGG16 apresentou bons resultados sem a utilização de transferência de aprendizado, o que sugere que essa base pode apresentar características mais fáceis de serem aprendidas pelos modelos.
Embora o autor não tenha especificado, uma análise visual das imagens dessa base indica que as fissuras presentes têm largura considerável, o que pode ter facilitado a detecção das mesmas pelos modelos treinados.
No entanto, como já comentado na \autoref{sub:bases}, é importante mencionar que há um problema relacionado ao número de imagens disponíveis na base. 
O número de imagens disponíveis para download e o número de imagens mencionado no artigo original são cerca de 4 imagens, podendo ter afetado os resultados, especialmente se essas imagens faltantes contivessem informações cruciais.
Em diversos casos, os modelos apresentaram uma acurácia de 99,9\% ou superior, contudo, é possível que essas mesmas imagens, erroneamente rotuladas, tenham impedido o modelo de atingir a marca de 100\%.
Por fim, a análise visual também permite perceber que a textura de fundo não é sempre a mesma, tendo variação considerável entre textura lisa, áspera e com marcas.
Contudo, essas imagens não possuem elementos que o modelo poderia confundir com as fissuras.


A base de \citeonline{xu2019automatic} obteve, de modo geral, excelentes resultados em relação ao $F_{1}-Score$ e acurácia, sempre superiores a 98\%. 
No entanto, é importante mencionar que sua média de perda variou de 1\% a 5\% dependendo do modelo, o que indica a possibilidade de melhoria de seus resultados. 
Observa-se, porém, que a maioria das fissuras têm larguras facilmente detectáveis ao analisar visualmente a base.
Da mesma forma que a base de \citeonline{zhang_base2018}, a análise visual permite observar variações consideráveis na textura de fundo, o que é benéfico para a generalização do modelo.
Entretanto, algumas fissuras de pequena largura podem ter sido perdidas durante a redução de resolução para a entrada do modelo. 

A base de \citeonline{zoubir2021crack} apresenta resultados de $F_{1}-Score$ entre 93\% e 95\%, e precisão de 94\% a 98\%, que são ótimo resultados mas que mostram que há grande quantidade de falsos positivos.
Além disso, esse modelo apresenta uma variação de \textit{loss} entre 5\% e 10\%, o que representa que independente do modelo, a base apresenta problemas de aprendizado.
Uma análise visual permite perceber que há uma variação muito grande de texturas de fundo da imagem, o que pode se tornar problemático dado a baixa quantidade de imagens da base.
Fora que, existem alguns casos em que a diferença entre um elemento da textura e uma fissura é de difícil interpretação.
Para mais, a base conta com uma alta variedade de tipos de fissuras.
Por conta disso, é sugerido que ao utilizar essa base, seja utilizada técnicas de \textit{data augmentation} para aumentar a quantidade de imagens.


A base SDNET2018, criada por \citeonline{maguire2018sdnet2018}, apresenta resultados controversos em relação ao desempenho do modelo. 
Os valores de $F_{1}-Score$ variam entre 68\% e 76\%, enquanto a precisão varia de 80\% a 86\%. 
Já a sensibilidade, ou \textit{recall}, varia de 58\% a 68\%. 
Essa variação nos resultados pode ser atribuída a diversos fatores, como a qualidade das imagens, a variação na textura do fundo, o desbalanceamento de classes e a presença de ruídos.
No geral, a base SDNET2018 é uma das mais completas em termos de quantidade de imagens e variedade de características apresentadas.
Isso faz com que os modelos treinados com essa base sejam mais genéricos e capazes de lidar com uma maior diversidade de situações.
Por outro lado, isso também faz com que os modelos sejam mais difíceis de serem treinados, requerendo a utilização de mais técnicas, como \textit{data augmentation}, e escolha mais precisa de seus parâmetros e hiperparâmetros.
Também vale citar que dentre as bases utilizadas, essa é a base que apresenta imagens com maior resolução, logo, sendo a mais prejudicada por conta da redução de resolução.
Como os autores \citeonline{maguire2018sdnet2018} descrevem, há fissuras de 6 milímetros de largura, tendo grande chance destas sumirem durante esta redução.
De forma geral, pela quantidade de problemas, os resultados foram satisfatórios.

Por fim, a base 'Subconjunto 40k', que novamente, é um subconjunto de todas as imagens, obteve ótimos resultados, com valores de $F_{1}-Score$ variando entre 92\% e 94\%.
Contudo, obteve valores de \textit{loss} de 13\% a 27\%, que representam um desempenho abaixo do esperado e indicam que o modelo ainda tem capacidade de ser aprimorado.
No geral, por conter imagens de todas as outras bases, parte dos problemas presentes nessas bases viram problemas nesta também.
Por outro lado, essa é uma base útil por ser grande e diversificada e que cobre uma ampla gama de situações e condições.
Por conta disso, seus resultados são considerados satisfatórios.

\section{Comparação da aplicação das bases em outras bases}

% As Tabelas \ref{tab:zhang}, \ref{tab:sdnet2018}, \ref{tab:hajar}, \ref{tab:xu} e \ref{tab:subs}

As Tabelas de \ref{tab:zhang} a \ref{tab:subs} apresentam um comparativo dos resultados obtidos ao testar modelos treinados em uma base de dados em outras bases, incluindo a união dessas bases.
O método de análise utilizado foi o $F_{1}$-Score, e todos os resultados apresentados nessas tabelas correspondem aos cálculos desse indicador.

\subsection{Treinamento na base de \citeonline{zhang_base2018}}

\begin{table}[htb]
\centering
\caption{Resultados dos modelos treinados na base de \citeonline{zhang_base2018} testados em outras bases.}
\label{tab:zhang}
\begin{tabular}{|l|p{2.5cm}|p{2.5cm}|p{2.5cm}|p{2.5cm}|}
\hline
\diagbox[]{Modelo\\utilizado}{Base\\testada} & \citeonline{maguire2018sdnet2018} & \citeonline{zoubir2021crack} & \citeonline{xu2019automatic} & União das três bases. \\ \hline \hline
VGG16 sem TL & 22.05\% & 46.49\% & 83.99\% & 35.41\% \\ \hline
VGG16 com TL & 35.99\% & 83.99\% & 97.06\% & 59.77\% \\ \hline
Densenet & 32.08\% & 80.80\% & 90.75\% & 57.31\% \\ \hline
Resnet & 38.73\% & 80.36\% & 93.24\% & 59.94\% \\ \hline
\end{tabular}
\fdadospesquisa
\end{table}

Os resultados apresentados na \autoref{tab:zhang} indicam uma falta de capacidade da base de \citeonline{zhang_base2018} para generalizar os dados. 
Isso é oposto aos resultados obtidos em validação cruzada estratificada e testes, onde a base apresentou resultados ótimos quando usada apenas em sua própria base de dados.

Esses resultados ajudam a fundamentar a análise visual apresentada na \autoref{sub:compa}, já que, como mencionado, as fissuras na base de dados são todas muito visíveis e largas. 
Portanto, quando aplicados em outras bases que possuem fissuras menores, os modelos treinados nessa base não apresentam resultados satisfatórios. 
A exceção são os resultados de \citeonline{xu2019automatic}, o que pode indicar que ambas as bases possuem características similares.

Ao realizar a média aritmética de cada linha da \autoref{tab:zhang}, é possível obter a média da performance de cada modelo.
Os modelos VGG16 sem transferência de aprendizado, VGG16 com transferência de aprendizado, Densenet e Resnet apresentaram, em média, valores de $F_{1}$-Score de 46,98\% 69,20\% 65,23\% e 68,07\%, respectivamente.
Dessa forma, o modelo VGG16 com a transferência de aprendizado obteve a melhor média de desempenho, seguido por Resnet.
É importante ressaltar que o modelo VGG16 sem transferência de aprendizado apresentou aprendizado quando treinado na base de \citeonline{zhang_base2018}, no entanto, os resultados foram significativamente inferiores aos modelos que utilizam a transferência de aprendizado.

\subsection{Treinamento na base de \citeonline{maguire2018sdnet2018}}

\begin{table}[htb]
\centering
\caption{Resultados dos modelos treinados na base SDNET2018 de \citeonline{maguire2018sdnet2018} testados em outras bases.}
\label{tab:sdnet2018}
\begin{tabular}{|l|p{2.5cm}|p{2.5cm}|p{2.5cm}|p{2.5cm}|}
\hline
\diagbox[]{Modelo\\utilizado}{Base\\testada} & \citeonline{zhang_base2018} & \citeonline{zoubir2021crack} & \citeonline{xu2019automatic} & União das três bases. \\ \hline \hline
VGG16 sem TL & 0.00\% & 0.00\% & 0.00\% & 0.00\% \\ \hline
VGG16 com TL & 84.00\% & 78.40\% & 96.13\% & 85.68\% \\ \hline
Densenet & 87.55\% & 83.74\% & 95.49\% & 88.72\% \\ \hline
Resnet & 82.68\% & 85.61\% & 94.03\% & 84.80\% \\ \hline
\end{tabular}
\fdadospesquisa
\end{table}

Ao contrário dos resultados obtidos na validação cruzada estratificada e nos testes, em que a base apresentou resultados duvidosos, na \autoref{tab:sdnet2018} é possível observar que os modelos treinados na base de \citeonline{maguire2018sdnet2018} conseguiram realizar uma boa abstração das características necessárias para identificar fissuras. 
Esses resultados destacam, mais uma vez, por que o SDNET2018 é reconhecido no campo acadêmico.

Nesse caso, as médias obtidas pelos modelos foram de 0,0\%, 86,05\%, 88,87\% e 86,78\%. 
Portanto, o modelo que obteve melhor desempenho foi o Densenet, seguido pelo Resnet.

\subsection{Treinamento na base de \citeonline{zoubir2021crack}}
\begin{table}[htb]
\centering
\caption{Resultados dos modelos treinados na base de \citeonline{zoubir2021crack} testados em outras bases.}
\label{tab:hajar}
\begin{tabular}{|l|p{2.5cm}|p{2.5cm}|p{2.5cm}|p{2.5cm}|}
\hline
\diagbox[]{Modelo\\utilizado}{Base\\testada} & \citeonline{zhang_base2018} & \citeonline{maguire2018sdnet2018} & \citeonline{xu2019automatic} & União das três bases. \\ \hline \hline
VGG16 sem TL & 0.00\% & 0.00\% & 0.00\% & 0.00\% \\ \hline
VGG16 com TL & 83.88\% & 39.32\% & 92.36\% & 73.61\% \\ \hline
Densenet & 69.59\% & 33.17\% & 85.38\% & 64.01\% \\ \hline
Resnet & 87.40\% & 36.73\% & 88.13\% & 77.02\% \\ \hline
\end{tabular}
\fdadospesquisa
\end{table}

Os resultados apresentados na \autoref{tab:hajar} demonstram resultados insatisfatórios.
Logo, pode se definir que a base de \citeonline{zoubir2021crack} por si só não é uma boa alternativa para treinamento de modelos de rede neurais caso o objetivo seja detectar de forma geral imagens de fissuras.

É importante destacar, no entanto, que essa conclusão não invalida o potencial da base de dados para aplicações específicas, bem como para o uso em conjunto com outras bases de dados para um melhor desempenho de modelos de redes neurais voltados para a detecção de imagens de fissuras.
Além disso, é possível que o uso de técnicas de \textit{data augmentation} possa tornar essa base mais robusta e melhorar sua capacidade de generalização para a detecção de fissuras em imagens. 
A aplicação de técnicas de \textit{data augmentation} pode ajudar a aumentar a variabilidade dos dados de treinamento e, assim, melhorar o desempenho de modelos de redes neurais treinados com essa base.

Quando treinados nessa base de dados e testado nas outras bases de dados, os modelos testados tiveram em média, valores de $F_{1}$-Score 0,0\%, 72,29\%, 63,04\% e 72,32\%..
Assim, o modelo Resnet obteve os melhores resultados, seguido pelo VGG16 com transferência de aprendizado.

\subsection{Treinamento na base de \citeonline{xu2019automatic}}
\begin{table}[htb]
\centering
\caption{Resultados dos modelos treinados na base de \citeonline{xu2019automatic} testados em outras bases.}
\label{tab:xu}
\begin{tabular}{|l|p{2.5cm}|p{2.5cm}|p{2.5cm}|p{2.5cm}|}
\hline
\diagbox[]{Modelo\\utilizado}{Base\\testada} & \citeonline{zhang_base2018} & \citeonline{maguire2018sdnet2018} & \citeonline{zoubir2021crack} & União das três bases. \\ \hline \hline
VGG16 sem TL & 66.67\% & 26.28\% & 43.10\% & 45.63\% \\ \hline
VGG16 com TL & 97.15\% & 39.60\% & 72.43\% & 77.15\% \\ \hline
Densenet & 96.04\% & 43.84\% & 79.33\% & 81.59\% \\ \hline
Resnet & 96.58\% & 36.83\% & 70.54\% & 74.18\% \\ \hline
\end{tabular}
\fdadospesquisa
\end{table}

A partir dos resultados presentes na \autoref{tab:xu}, pode-se concluir que a base de \citeonline{xu2019automatic} não apresenta todas as características necessárias para que os modelos treinados nela sejam capazes de detectar com boa precisão imagens de fissuras.
Vale ressaltar que essa base tem bons resultados apenas na base de \citeonline{zhang_base2018}, evidenciando que ambas as bases possuem características semelhantes e, portanto, podem complementar uma à outra para um melhor desempenho na detecção de fissuras em imagens.

As médias de $F_{1}$-Score obtidas pelos modelos quando treinados na base de \citeonline{xu2019automatic} foram de 45,42\%, 71,58\%, 75,20\% e 69,53\%.
Assim sendo, o modelo que obteve melhor desempenho foi o Densenet, seguido pelo VGG16 com transferência de aprendizado.


\subsection{Treinamento na base 'Subconjunto 40k'}
\begin{table}[htb]
\centering
\caption{Resultados dos modelos treinados na base 'Subconjunto 40k' testados em outras bases.}
\label{tab:subs}
\begin{tabular}{|l|p{2.2cm}|p{2.2cm}|p{2.2cm}|p{2.2cm}|p{2.2cm}|}
\hline
\diagbox[width=8em]{Modelo\\utilizado}{Base\\testada} & \citeonline{zhang_base2018} & \citeonline{maguire2018sdnet2018} & \citeonline{zoubir2021crack} & \citeonline{xu2019automatic} & 69K imagens \\ \hline \hline
VGG16 sem TL & 98.77\% & 0.00\% & 0.00\% & 71.31\% & 78.00\% \\ \hline
VGG16 com TL & 99.97\% & 78.97\% & 97.85\% & 99.92\% & 94.96\% \\ \hline
Densenet & 99.98\% & 85.39\% & 99.12\% & 99.92\% & 96.46\% \\ \hline
Resnet & 99.95\% & 86.95\% & 98.83\% & 99.71\% & 96.78\% \\ \hline
\end{tabular}
\fdadospesquisa
\end{table}

Conforme já definido anteriormente, a base de dados 'Subconjunto 40k' é um conjunto balanceado de 40.000 imagens selecionadas aleatoriamente do conjunto total de imagens. 
Ao usar modelos treinados nessa base para testes em outras bases, foi necessário ter cuidado para não selecionar os mesmos dados usados para a validação cruzada nos testes. 
Como 90\% das imagens foram usadas na validação cruzada, existem 36.000 imagens que não devem ser usadas nos testes. 
Logo, ao remover as 36.000 imagens usadas na validação cruzada do total de 105.984 imagens, sobram
69.984 imagens.
Por esse motivo, a última coluna da \autoref{tab:subs} é denominada '69K imagens'. 

Essa remoção de certas imagens das bases de dados gera uma variação na comparação do resultado presente na \autoref{tab:subs}, com os outros resultados.
Entretanto, essa variação tem um impacto mínimo na comparação deste com outros resultados nesta seção, mas que deve ser mencionado.

Os modelos quando treinados na base 'Subconjunto 40k', obtiveram ótimos resultados, como pode ser observado na \autoref{tab:subs}, onde a maioria dos resultados são valores de $F_{1}$-Score maiores que 90\%.
O fato dessa base conter imagens de todas as bases faz com que apresente um maior número de características para serem abstraídas, e caso sejam, o modelo obterá uma maior capacidade de abstração.

Conforme pode ser observado na \autoref{tab:subs}, os modelos treinados na base de dados 'Subconjunto 40k' obtiveram excelentes resultados, com a maioria dos valores de $F_{1}$-Score acima de 90\%. 
Isso se deve ao fato de que essa base de dados contém imagens de todas as bases, o que proporciona um maior número de características a serem abstraídas pelo modelo.
Caso o modelo utilizado consiga fazer a abstração desses dados, após seu treinamento, ele terá uma melhor capacidade de generalização para imagens de outras bases.

Com base nos resultados obtidos até o momento, conclui-se que o método de treinamento mais eficaz foi o uso de um subconjunto que abrangesse todas as bases de dados. 
No entanto, é importante ressaltar que foi necessário retirar um certo número de imagens de cada base para evitar a repetição de imagens no treinamento do modelo. 
Embora essa exclusão possa ter um efeito mínimo, a diferença nos resultados obtidos nos testes de cada base é muito significativa, o que sugere que esse fator não influenciou significativamente os resultados finais.

Além disso, em média, os modelos VGG16 sem transferência de aprendizado, VGG16 com transferência de aprendizado, Densenet e Resnet apresentaram valores de $F_{1}$-Score de 49,62\%, 94,33\%, 96,17\% e 96,44\%, respectivamente.
Portanto, é possível afirmar que o modelo Resnet obteve o melhor desempenho, demonstrando uma maior capacidade de abstração e generalização de dados em comparação aos outros modelos avaliados.

\chapter{Cronograma} %Futuras conclusões
\label{chapter:conclusao}

Em geral, os modelos treinados durante a validação cruzada estratificada \textit{K-fold} que não empregaram a técnica de transferência de aprendizado não foram capazes de aprender a classificar imagens como fissuradas ou saudáveis, resultando na incapacidade de detectar fissuras. 
Mesmo aqueles modelos que conseguiram aprender sem a técnica de transferência de aprendizado foram superados por suas contrapartes que a utilizaram.

Em contraste, os modelos que empregaram a técnica de transferência de aprendizado com pesos provenientes do conjunto de dados \textit{Imagenet} obtiveram resultados positivos, com acurácia superior a 90\% em todos os casos e valores de $F_{1}-Score$ superiores a 90\% na maioria dos casos.
Além disso, quando testados em novas imagens dentro do mesmo conjunto de dados, esses modelos continuaram a apresentar resultados similares, o que, combinado com o uso do \textit{K-fold}, demonstra a confiabilidade desses modelos.

Os modelos utilizados, VGG16, Densenet e Resnet obtiveram nos testes, médias de de $F_{1}-Score$ de 92,93\%, 91,11\% e 91,12\%, respectivamente.
Logo, o modelo VGG16 obteve a melhor média de desempenho em comparação aos outros modelos avaliados.
Em seguida, o modelo Resnet apresenta a segunda melhor média, e por fim, o Densenet.

Adicionalmente, esta monografia oferece informações relevantes de cada fonte de dados através de análises visuais meticulosas. 
Embora não sejam necessariamente definições estabelecidas, esses argumentos podem ser utilizados como um auxílio valioso e um contribuinte significativo para o tema em questão. 
Através de uma análise visual aprofundada, a monografia apresenta uma compreensão mais clara e concisa das informações obtidas a partir de cada fonte de dados, permitindo que o leitor compreenda melhor as nuances do tópico discutido.
Com base nisso, é possível extrair informações valiosas e utilizá-las para embasar ainda mais a discussão em torno do tema.

Além disso, foram realizados experimentos para avaliar o desempenho de modelos treinados em uma base de dados e testados em outras. 
Esses experimentos complementam  as análises visuais e evidenciaram que o uso de um subconjunto que abrange as imagens de todas as bases de dados é o método de treinamento mais eficaz. 
Cabe destacar que o modelo Resnet apresentou o melhor desempenho nos testes realizados com essa base, indicando sua maior capacidade de abstração e generalização de dados em comparação aos outros modelos avaliados.

Em resumo, as redes neurais convolucionais são ferramentas excelentes e de baixo custo que produzem resultados satisfatórios. 
No entanto, ainda é necessário um especialista para selecionar, treinar, testar e implementar um modelo. 
Além disso, embora os resultados possam ser muito bons, a intervenção humana é inevitável, pois qualquer erro pode ter consequências graves e irreversíveis. 

% ---
% Finaliza a parte no bookmark do PDF, para que se inicie o bookmark na raiz
% ---
\bookmarksetup{startatroot}% 
% ---

% ----------------------------------------------------------
% ELEMENTOS PÓS-TEXTUAIS
% ----------------------------------------------------------
\postextual

% ----------------------------------------------------------
% Referências bibliográficas
% ----------------------------------------------------------
\bibliography{references}

% ---------------------------------------------------------------------
% GLOSSÁRIO
% ---------------------------------------------------------------------

% Arquivo que contém as definições que vão aparecer no glossário
\input{tex/glossario}
% Comando para incluir todas as definições do arquivo glossario.tex
\glsaddall
% Impressão do glossário
\printglossaries

% ----------------------------------------------------------
% Apêndices
% ----------------------------------------------------------

% ---
% Inicia os apêndices
% ---
%\begin{apendicesenv}
%
%    \chapter{}
%    \label{}
%    \input{}
%
%\end{apendicesenv}
% ---


% ----------------------------------------------------------
% Anexos
% ----------------------------------------------------------

% ---
% Inicia os anexos
% ---
%\begin{anexosenv}
%
%    \chapter{} 
%    \label{}
%    \input{tex/annex/}
%
%\end{anexosenv}
% ---

\end{document}